\documentclass[10pt]{article}
\usepackage{minted}
\usepackage{mathtools}
\usepackage{amsfonts}
\usepackage{pifont}
\usepackage[shortlabels]{enumitem}
\newcommand*{\perm}[2]{{}^{#1}\!P_{#2}}%
\author{BM Corser}
\title{Statistics, Assignment 2}
\begin{document}
    \maketitle 
    \begin{enumerate}
        \item
            \begin{enumerate}[(i)]
                \item If the die is fair, and that its rolls follow a normal
                    distribution, the population mean will be $\mu_0 =
                    \frac{7}{2}$. We can use a one sample $t$-test in which our
                    null hypothesis is that the die is fair and our alternative
                    hypothesis is that the die is not fair.  As such $H_0: \mu
                    = \mu_0$ and $H_1: \mu \neq \mu_0$. Our alternative
                    hypothesis is two-sided. Our test statistic is $t =
                    \left|\frac{\bar{x} - \mu_0}{\frac{s}{\sqrt{n}}}\right|$
                    and we will reject our null hypothesis $H_0$ at the
                    $100\alpha \%$ significance level if the percentage point
                    $t_{n-1}(50\alpha) < t$.
                \item $\bar{x} =
                    \frac{17}{5}$, $s = \sqrt{\frac{544112}{999}}$,
                    \begin{align*}
                        t &= \left|\frac{\bar{x} - \mu_0}{\frac{s}{\sqrt{n}}}\right| \\
                          &= \left|\frac{\frac{17}{5} - \frac{7}{2}}{\sqrt{\frac{\frac{544112}{999}}{1000}}}\right| \\
                          &= 3\frac{\sqrt{\frac{555}{68014}}}{2} \\
                          &= 1.81668...
                    \end{align*}
                    Using Table 10. from NCST to find $t_\nu(P)$ with $\nu =
                    \infty$ and $P = 2.5$ (because our $H_1$ is two sided)  as
                    an estimate for $t_{999}(2.5)$ we find that $t_\infty(2.5)
                    = 1.96$ which is greater than our test statistic, $t$,
                    above. Therefore we don't reject $H_0$ at the 5\%
                    significance level, and the die is probably fair.
                    \pagebreak
                \item
                    R input
                    \begin{minted}{r}
scores = c(
    rep(1, times = 212),
    rep(2, times = 140),
    rep(3, times = 156),
    rep(4, times = 170),
    rep(5, times = 172),
    rep(6, times = 150)
)
t.test(scores, mu=3.5, alternative="two.sided")
                    \end{minted}
                    R output
                    \begin{minted}{text}

	One Sample t-test

data:  scores
t = -1.814, df = 999, p-value = 0.06998
alternative hypothesis: true mean is not equal to 3.5
95 percent confidence interval:
 3.291821 3.508179
sample estimates:
mean of x 
      3.4 
                    \end{minted}
            \end{enumerate}
        \item
            \begin{enumerate}[(i)]
                \item Assuming the proportion of individuals having a
                    particular characteristic follows a binomial distribution,
                    we can test the null hypothesis $H_0: p = 0.5$ against the
                    one-sided alternative hypothesis $H_1: p > 0.5$ by using
                    the test statistic $z = \frac{\hat{p}-0.5}{\sqrt{\frac{(0.5)^2}{5008}}}$. We reject
                    $H_0$ at the 100$\alpha$\% significance level if the
                    percentage point of a normal distribution $x(100\alpha) < z$.
                \item $\hat{p} = \frac{2524 + 896}{5008} = \frac{3420}{5008}$,
                    $z = \frac{\frac{3420}{5008} - 0.5}{\sqrt{\frac{0.25}{5008}}} = 24.877$, $\alpha = 0.05$.
                    Since $x(100\alpha) = 1.6449 < z$, we reject $H_0$ at the
                    5\% significance level. There is strong evidence that there
                    is an association between between eye colour of parents and
                    children.
                \item R input
                    \begin{minted}{r}
prop.test(3420,5008,alternative="greater",correct=FALSE)
                    \end{minted}
                    R output
                    \begin{minted}{text}
	1-sample proportions test without continuity correction

data:  3420 out of 5008, null probability 0.5
X-squared = 670.17, df = 1, p-value < 2.2e-16
alternative hypothesis: true p is greater than 0.5
95 percent confidence interval:
 0.671995 1.000000
sample estimates:
        p 
0.6829073 
                    \end{minted}
            \end{enumerate}
        \item
            \begin{enumerate}[(i)]
                \item Assuming the difference follows a normal distribution, $D
                    \sim N(\mu_D, \sigma^2_D)$ can use a paired comparison with
                    null hypothesis $H_0: \mu_D = 0$ and the one-sided
                    alternative hypothesis $H_1 : \mu_D < 0$. We can use the
                    test statistic $t = \frac{\bar{d}}{\frac{s_D}{\sqrt{n}}}$
                    where $n = 9$, $\bar{d} = 8.443333$, $s^2_D = 981.8365$. Then
                    $t = 0.80838$ and the percentage point for a
                    $t$-distribution with $\nu = 9$ at 5 is $t_9(5) = 1.833$.
                    Because $t < t_9(5)$ we do not reject $H_0$ at the 5\%
                    significance level and as such there is no strong evidence
                    that the drug is effective at reducing cholesterol.

                    R input
                    \begin{minted}{r}
NoDrug = c(206.13, 203.62, 226.43, 139.81, 137.40, 131.80, 145.41, 141.64, 216.86)
Drug = c(166.81, 181.14, 211.65, 96.99, 141.41, 166.91, 101.25, 169.05, 237.90)
D = NoDrug - Drug
t.test(D)
                    \end{minted}

                    R output
                    \begin{minted}{text}
	One Sample t-test

data:  D
t = 0.80838, df = 8, p-value = 0.4422
alternative hypothesis: true mean is not equal to 0
95 percent confidence interval:
 -15.64232  32.52899
sample estimates:
mean of x 
 8.443333 
                    \end{minted}
                \item Observing our variable \mintinline{text}{D}, we can see
                    that $n = 9$ (since no difference is 0) and there are four
                    negative differences.

                    R output
                    \begin{minted}{text}
> D
[1]  39.32  22.48  14.78  42.82  -4.01 -35.11  44.16 -27.41 -21.04
                    \end{minted}

                    We can then set up a sign test against a null hypothesis
                    $H_0: \eta = 0$ and one sided alternative hypothesis $H_1:
                    \eta < 0$

                    Referring to Table 1 with $n = 9$, $r = 4$ and $p = 0.5$ we
                    can find a $p$-value st. $p = \Pr(X \leq 4) = F_3 = 0.5$.
                    Because $p > 0.1$ we do not reject $H_0$ at the 10\%
                    significance level.

                    \pagebreak

                    R input
                    \begin{minted}{r}
binom.test(sum(D < 0), sum(D != 0), alternative="less")
                    \end{minted}

                    R output
                    \begin{minted}{text}
	Exact binomial test

data:  sum(D < 0) and sum(D != 0)
number of successes = 4, number of trials = 9, p-value = 0.5
alternative hypothesis: true probability of success is less than 0.5
95 percent confidence interval:
 0.0000000 0.7486324
sample estimates:
probability of success
             0.4444444
                    \end{minted}
                \item 
                    $H_0 : \eta = 0$, $H_1: \eta < 0$

    \begin{tabular}{ c c }
        $d_i$ & signed rank \\
        \hline
        39.32 & 7 \\
        22.48 & 4 \\
        14.78 & 2 \\
        42.82 & 8 \\
        -4.01 & -1 \\
        -35.11 & -6 \\
        44.16 & 9 \\
        -27.41 & -5 \\
        -21.04 & -3 \\
    \end{tabular}

    $T^+ = 30$, $x(P) = 8$ and since $x(P) < T^+$, we do not reject $H_0$ at a
    5\% significance level; there is some evidence that the drug is effective
    at reducing cholesterol.

                    \begin{minted}{r}

wilcox.test(D)
                    \end{minted}

                    R output
                    \begin{minted}{text}
	Wilcoxon signed rank test

data:  D
V = 30, p-value = 0.4258
alternative hypothesis: true location is not equal to 0
                    \end{minted}
            \end{enumerate}
    \end{enumerate}
\end{document}
