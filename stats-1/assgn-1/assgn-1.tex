\documentclass[10pt]{article}
\usepackage{listings}
\usepackage{mathtools}
\usepackage{amsfonts}
\usepackage{pifont}
\usepackage[shortlabels]{enumitem}
\newcommand*{\perm}[2]{{}^{#1}\!P_{#2}}%
\author{BM Corser}
\title{Statistics Assignment 2}
\begin{document}
    \maketitle 
    \begin{enumerate}
        \item Let $R = (280, 155, 329, 140, 307, 116, 202, 262, 130, 131, 187, 187, 292,
            83, 207, 197, 134, 294, 163, 217)$
            \begin{enumerate}[(i)]
            \item 
                \begin{align*}
                    \bar{x} &= \frac{1}{n}\cdot\sum_{i=0}^n R_i \\
                    &= \frac{1}{20}\cdot(280 + 155 + 329 + 140 + 307 + 116 + 202 + 262 + 130 + \ \dots + R_n) \\
                    &= 200.65
                \end{align*}
            \item 
                \begin{align*}
                    \sigma^2 &= \frac{1}{n - 1}\cdot\sum_{i=0}^n (\bar{x} - r)^2 \\
                    &= \frac{1}{19}\cdot((200.65 - 280)^2 + (200.65 - 155)^2 + \ \dots) \\
                    &= \frac{1}{19}\cdot((-79.35)^2 + 45.65^2 + \ \dots) \\
                    &\approx 5161.60789
                \end{align*}
            \item 
                Python code:
                \lstinputlisting{1.py}
                Output:
                \lstinputlisting{1.out}
            \end{enumerate}
        \item
            \begin{enumerate}[(i)]
                \item $\Pr(B^c) = 1 - \Pr(B) = 0.85$
                \item $\Pr(A \cap B) = \Pr(B) \cdot \Pr(A|B) = 0.15 \times 0.67 = 0.1139$
                \item $\Pr(A \cap B^c) = \Pr(A) - \Pr(A \cap B) = 0.3 - 0.1139 = 0.1861$
                \item $\Pr(A | B^c) = \frac{\Pr(A \cap B^c)}{\Pr(B^c)} = \frac{0.1861}{0.85} = 0.21894117647058822$
                \item The events are not independent, because $\Pr(A|B) \not= \Pr(A)$
                \item The events are not mutually exclusive, because $\Pr(A \cap B) \not= 0$
            \end{enumerate}
        \item 
            \begin{enumerate}[(i)]
                \item
                    $\binom{3000}{1} \cdot 0.002^1 \cdot 0.998^{2999} = 0.0148130528123 \dots$
                    \\

                    Python code:
                    \lstinputlisting{3i.py}
                    Output:
                    \lstinputlisting{3i.out}
                \item If an event happens $r$ times in $n$ samples the
                    probability it happens is $\frac{r}{n}$. In the case where
                    this event either happened an amount of 0 (ie. it didn't
                    happen) or an amount of 1 (ie.  it completely happened) for
                    a given sample, the mean is therefore equivalent to the
                    probability because $\frac{1}{n}(r \cdot 1 + (n - r)\cdot
                    0) = \frac{r}{n}$ so in this case we may use a Poisson
                    distribution with $\mu = 0.002$. Hence, according to the
                    Poisson estimation, the probabilty that one dog gets sick
                    is
                    $$
                        \frac{0.002^1}{e^{0.002}\cdot1!} = 0.00199600 \dots
                    $$
                    \\

                    Python code:
                    \lstinputlisting{3ii.py}
                    Output:
                    \lstinputlisting{3ii.out}
                \item I'm not entirely sure how to scale a Normal distribution
                    based on probability of a discrete random variable instead
                    of $\mu$ and $\sigma^2$ on continuous data, but going by
                    the summary notes the $\mu$ value from a Poisson
                    distribution can be used as $\mu$ and $\sigma^2$ in
                    $N(\mu,\sigma^2)$, hence let $\mu = \sigma^2 = 0.002$ in
                    $$
                        f(x) = \frac{1}{2\pi\sigma^2}\exp\left(-\frac{(x - \mu)^2}{2\sigma^2}\right)
                    $$
                    then
                    \begin{align*}
                        f(1) &= \frac{1}{0.004\pi}\exp\left(-\frac{(1 - 0.002)^2}{0.004}\right)\\
                         &= 5.768061039566294686832425143 ... \times 10^{-107}
                    \end{align*}
                    \\

                    Python code:
                    \lstinputlisting{3iii.py}
                    Output:
                    \lstinputlisting{3iii.out}
                    Where presumabley the float representation in the Python
                    library I used wasn't accurate enough to show this number.
            \end{enumerate}
        \item $\Pr(X \leq r)$ for a Poisson distribution with $\mu = 2$ is
            given by $$F_r = \sum_{i=0}^re^{-2}\frac{2^i}{i!},$$
            so evaulating for $r = 2$
                \begin{align*}
                    F_2 = \sum_{i=0}^2e^{-2}\frac{2^i}{i!} &= e^{-2}\frac{2^0}{0!} + e^{-2}\frac{2^1}{1!} + e^{-2}\frac{2^2}{2!} \\
                    &= e^{-2} + 2e^{-2} + 2e^{-2} \\
                    &= 5e^{-2}.
                \end{align*}
                As such, the probability two or fewer items will be sold in a
                day is $5e^{-2}$. The probability that $F_r$ will occur on 7
                successive is $(F_r)^7$, which evaluating for $r = 2$, as
                before, $\left(5e^{-2}\right)^7 = 5^7e^{-14} < 0.95$. We can
                write a Python program to try increasing values for $r$ until
                we get a result $(F_r)^7 > 0.95$.

                \pagebreak

                Python code:
                \lstinputlisting{4.py}
                Output:
                \lstinputlisting{4.out}

                Therefore, in order to have 96.8\% probability of not running
                out of stock, the salesperson will have to buy 7 days $\times$
                6 items = 42.

                Because the above assumes that the same number of items will
                need to be bought each day, it gives us a kind of ``upper
                bound''. Let's see if we can shave an item off and still
                achieve our desired probability.
                \\

                Python code:
                \lstinputlisting{42.py}
                Output:
                \lstinputlisting{42.out}

                We can see here that $(F_6)^6 \cdot F_5$ also results in a
                probability $> 0.95$, so the total number of items the
                salesperson needs to buy is $6 \times 6 + 5 = 41$.
            \item  $X \sim N(68, 10)$
            \begin{enumerate}[(i)]
                \item $1 - \Phi(75) = 0.241964$
                \item $\Phi(75) - \Phi(70) = 0.758036 - 0.57926 = 0.178776$
            \end{enumerate}
    \end{enumerate}
\end{document}
