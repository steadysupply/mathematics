\documentclass[10pt]{article}
\usepackage[shortlabels]{enumitem}
\usepackage{mathtools}
\usepackage{hyperref}
\usepackage{amsfonts}
\usepackage{amsthm}
\usepackage{listings}
\author{BM Corser}
\title{Algebra 2 Assignment 3}
\newcommand*{\E}{$e$}
\newcommand*{\A}{$\alpha$ }
\newcommand*{\As}{$\alpha^2$ }
\newcommand*{\Ac}{$\alpha^3$ }
\newcommand*{\B}{$\beta$ }
\newcommand*{\AB}{$\alpha\beta$ }
\newcommand*{\AsB}{$\alpha^2\beta$ }
\newcommand*{\AcB}{$\alpha^3\beta$ }
\newcommand*{\MMR}{$\mathcal{M}_2(\mathbb{R})$ }
\newcommand*{\MR}{\mathcal{M}_2(\mathbb{R})}
\begin{document}
  \maketitle 
  \begin{enumerate}
    \item
    \begin{enumerate}
      \item Let $A,B\in T$ with
        $A = \begin{pmatrix} a & 0 \\ b & a \\ \end{pmatrix}$
        and
        $B = \begin{pmatrix} c & 0 \\ d & c \\ \end{pmatrix}$.
        Using the subring criterion
        \begin{enumerate}[(i)]
          \item
            $
            A+B =
            \begin{pmatrix} a & 0 \\ b & a \\ \end{pmatrix}
            +
            \begin{pmatrix} c & 0 \\ d & c \\ \end{pmatrix}
            =
            \begin{pmatrix} a + c & 0 \\ b + d & a + c \\ \end{pmatrix} \in T,
            $
          \item
            $
            -A = \begin{pmatrix} -a & 0 \\ -b & -a \\ \end{pmatrix} \in T,
            $
          \item
            $
            AB =
            \begin{pmatrix} a & 0 \\ b & a \\ \end{pmatrix}
            \begin{pmatrix} c & 0 \\ d & c \\ \end{pmatrix}
            =
            \begin{pmatrix} ac & 0 \\ bc + ad & ac \\ \end{pmatrix} \in T.
            $
        \end{enumerate}
        Hence $T$ is a subring of \MMR.
        \item For $T$ to be an ideal of \MMR, for all $r \in \MR$ and $s \in T$
          it must hold that $rs \in T$ and $sr \in T$. Let's test this
            $$
            rs =
            \begin{pmatrix} a & b \\ c & d \\ \end{pmatrix}
            \begin{pmatrix} x & 0 \\ y & x \\ \end{pmatrix}
            =
            \begin{pmatrix} ax + yb & bx \\ cx + yd & dx \\ \end{pmatrix} \not\in T.
            $$
            As such, $T$ is not an ideal of \MMR.
        \item As shown in 1.(a)(iii) the product of any two elements of $T$ has
          the form
            $
            AB =
            \begin{psmallmatrix} a & 0 \\ b & a \\ \end{psmallmatrix}
            \begin{psmallmatrix} c & 0 \\ d & c \\ \end{psmallmatrix}
            =
            \begin{psmallmatrix} ac & 0 \\ bc + ad & ac \\ \end{psmallmatrix}.
            $
          As such the zero divisors of $T$ are those elements for which
          $ac = 0$ and $bc + ad = 0$ with at least one of $a$, $b$ nonzero and
          at least one of $c$, $d$ nonzero.

          Let $a = c = 0$ and $b$, $d$ be nonzero, now we have $AB = 0$.
          Neither of $A$ or $B$ are zero and we have that $A$ is a zero
          divisor.
          Therefore the zero divisors of $T$ are
          $\left\{\begin{psmallmatrix} 0 & 0 \\ x & 0 \\ \end{psmallmatrix} \in T : x > 0\right\}$.
            \pagebreak
        \item Let $A, B \in T$;
          \begin{enumerate}[(i)]
            \item $T$ is commutative, we have

            \begin{align*}
              AB &=
              \begin{pmatrix} a & 0 \\ b & a \\ \end{pmatrix}
              \begin{pmatrix} c & 0 \\ d & c \\ \end{pmatrix} \\
                &= \begin{pmatrix} ac & 0 \\ bc + ad & ac \\ \end{pmatrix} \\
                &= \begin{pmatrix} ca & 0 \\ da + cb & ca \\ \end{pmatrix} \\
              &=
              \begin{pmatrix} c & 0 \\ d & c \\ \end{pmatrix}
              \begin{pmatrix} a & 0 \\ b & a \\ \end{pmatrix} \\
              &= BA,
            \end{align*}
          \item $T$ is is a ring with identity, we have $I_2$ the $2\times 2$
              identity matrix in $T$ and therefore $I_2A = A = AI_2$ for all
              $A$ in $T$,
            \item $T$ is not a division ring. To see this, let $B$ be $A$'s
              inverse and write 
            $
            AB =
            \begin{psmallmatrix} ac & 0 \\ bc + ad & ac \\ \end{psmallmatrix}
            = I_2 =
            \begin{psmallmatrix} 1 & 0 \\ 0 & 1 \\ \end{psmallmatrix}
            $, giving
              $ac = 1$ and $bc + ad = 0$. These equations hold when
              $c = \tfrac{1}{a}$ and $d = -\tfrac{b}{a^2}$. Now suppose $a =
              0$. In this case $B$ is undefined and $A$ does not have an
              inverse. QED

              Lemma, zero divisors of $T$ have no inverse.
          \end{enumerate}
      \end{enumerate}
        \item 
          \begin{enumerate}
            \item
              \begin{align*}
                p(x) &= f(x) - xg(x) \\
                     &= -x^3 + x^2 - x + 1 \\
                q(x) &= g(x) + p(x) \\
                     &= 3x^2 + 3 \\
                r(x) &= p(x) + \tfrac{1}{3}xq(x) \\
                     &= x^2 + 1 \\
                s(x) &= q(x) - 3r(x) \\
                     &= 0
              \end{align*}
              Hence $\text{gcd}(f(x),g(x)) = r(x) = x^2 + 1$.
            \item By long division we know that, $\tfrac{f(x)}{x^2 +1} = x^2 +
              x + 1$ and $f(x) = (x^2 + x + 1)(x^2 + 1)$. These two quadratic
              factors are irreducible because they have complex roots and
              therefore factors of the form $(x - z)$ where $z \in \mathbb{C}$
              so $(x - z) \not \in \mathbb{R}[x]$.
          \end{enumerate}
          \pagebreak
        \item
          \begin{enumerate}
            \item
              \begin{proof}
                If $a$ is a zero divisor of $R$, there must be some element
                $z \in R$ such that $az = 0$. Since $a$ is nilpotent, we know
                there exists some positive integer $n$ such that $a^n = 0$ Let
                $z = a^{n-1}$, it is easy to see that $az=aa^{n-1}=a^n=0$,
                therefore $a$ is a nilpotent element of $R$ if and only if $a$
                is a zero divisor of $R$.
              \end{proof}
            \item Clearly, 0 is a nilpotent element of $\mathbb{Z}_{12}$
              because $0^n = 0$ for all $n$. 6 is also a nilpotent element of
              $\mathbb{Z}_{12}$, because $6^2=36=0$. By inspection, there are
              no other nilpotent elements of $\mathbb{Z}_{12}$.
            \item
              \begin{proof}
                $0^n = 0$, so certainly $0$ is a nilpotent element of
                $\mathbb{Z}_{967}$. Let $a \not= 0$ be a nilpotent element
                of $\mathbb{Z}_{967}$, by the proof in 3.(a), $a$ is a zero
                divisor. Because 967 is prime, by Lemma 3.3.4,
                $\mathbb{Z}_{967}$ is a field and therefore every element is a
                unit. Now by Lemma 3.3.12 we have a contradiction. Hence 0 is
                the only nilpotent element $\mathbb{Z}_{967}$.
              \end{proof}
              % Because 967 is prime, no product of nonzero elements in
              % $\mathbb{Z}_{12}$ will equal zero. Certainly then, no power of
              % a single nonzero element will equal zero.
          \end{enumerate}
    \end{enumerate}
\end{document}
