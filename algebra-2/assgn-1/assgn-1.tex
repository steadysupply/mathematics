\documentclass[10pt]{article}
\usepackage{mathtools}
\usepackage{amsfonts}
\author{BM Corser}
\title{Algebra 2 Assignment 1}
\newcommand*{\E}{$e$}
\newcommand*{\A}{$\alpha$ }
\newcommand*{\As}{$\alpha^2$ }
\newcommand*{\Ac}{$\alpha^3$ }
\newcommand*{\B}{$\beta$ }
\newcommand*{\AB}{$\alpha\beta$ }
\newcommand*{\AsB}{$\alpha^2\beta$ }
\newcommand*{\AcB}{$\alpha^3\beta$ }
\begin{document}
  \maketitle 
  \begin{enumerate}
    \item
    \begin{enumerate}
        \item Let \A represent a clockwise rotation of $\frac{\pi}{2}$ about
          the origin and $\beta$ represent a particular reflection. Now the
          Cayley table of ${\rm Dih}(8)$ can be written

        \begin{tabular}{r|cccccccc}
               & \E   & \A   & \As  & \Ac  & \B   & \AB  & \AsB & \AcB \\ \hline
          \E   & \E   & \A   & \As  & \Ac  & \B   & \AB  & \AsB & \AcB \\
          \A   & \A   & \As  & \Ac  & \E   & \AB  & \AsB & \AcB & \B   \\
          \As  & \As  & \Ac  & \E   & \A   & \AsB & \AcB & \B   & \AB  \\
          \Ac  & \Ac  & \E   & \A   & \As  & \AcB & \B   & \AB  & \AsB \\
          \B   & \B   & \AcB & \AsB & \AB  & \E   & \Ac  & \As  & \A   \\
          \AB  & \AB  & \B   & \AcB & \AsB & \A   & \E   & \Ac  & \As  \\
          \AsB & \AsB & \AB  & \B   & \AcB & \As  & \A   & \E   & \Ac  \\
          \AcB & \AcB & \AsB & \AB  & \B   & \Ac  & \As  & \A   & \E   \\
        \end{tabular}
      \item The set of elements that commute in Dih(8) is $\{\alpha^2, e\}$
        this can be observed in the Cayley table above by noticing that for
        each of these elements, both the row and column for that element
        contain the same elements in the same order.
      \item The subgroups of Dih(8) are the trivial and identity subgroups
        $\{e\}$ and Dih(8), the four 2-subgroups formed by every reflecting
        element and the identity, $\{e, \alpha^n\beta\}$ for $0 \leq n \leq 3$,
        the commutative 2-subgroup $\{e, a^2\}$, the 4-subgroup containing the
        identity and all rotation elements $\{e, \alpha, \alpha^2, \alpha^3\}$
        and finally the 4-subgroup containing $\{e, \alpha^2, \alpha\beta,
        \alpha^3\beta\}$.

        By Lagrange, the order of any subgroup $H \leq {\rm Dih}(8)$, $|H|$
        must divide $|{\rm Dih}(8)|$. So there aren't any 3- or 5-subgroups
        because neither 3 or 5 divide 8.

        Now, a subgroup must use a subset of Dih(8) and we needn't consider
        subsets of size 3 or 5.

        There can only be one 1-subgroup, $\{e\}$, so there aren't any more of
        those.

        There aren't any 2-subgroups beyond those already mentioned, because
        none of the non-reflecting 2-subsets $\{e, \alpha^n\}$ with $1 \leq n
        \leq 3$ have closure.

        There aren't any more 4-subgroups because (apart from the last subgroup
        identified above) any subgroup that has a combination of reflections
        and rotations doesn't have closure.

        There can only be one 8-subgroup, Dih(8), so there aren't any more of
        those either.
    \end{enumerate}
  \item Let $k, l \in K$ and $1_K$ be the identity element of $K$.
    Since $K \leq G$, we know
    \begin{align*}
      k, l, kl \in G.
    \end{align*}
    Because $f : G \rightarrow H$, certainly 
      $f(k), f(l), f(kl) \in H$ and $f(K) \subseteq H$.

    Since $f$ is a homomorphism, we know that $f(ab) = f(a)f(b)$ and therefore
    \begin{align}
      f(k)f(l) \in H,
    \end{align}
    and, if $1_H$ is the identity element of $H$,
    \begin{align}
      f(1_K) = 1_H.
    \end{align}
          The fact that $f$ is a homomorphism also tells us that $f(a^{-1}) =
          f(a)^{-1}$ and since $k^{-1} \in K$ it is also true that
    \begin{align}
      f(k)^{-1} \in H.
    \end{align}
    Due to the facts that $K \leq G$ and that $f$ is a homomorphism, it is also
    true that if $j \in K$,
      \begin{align}
        f((jk)l) = f(jk)f(l) = f(j)f(k)f(l) = f(j)f(kl) = f(j(kl)).
      \end{align}

      Therefore $f(K)$ is closed (1), has an identity element $1_H$ (2), has
      inverses (3) and is associative (4). Hence, $f(K) \leq H$.
    \item 
      \begin{enumerate}
        \item $V_4 = \{e, a, b, c\}$ such that

          $ab = ba = c$,

          $ac = ca = b$,

          $bc = cb = a$ and

          $a^2 = b^2 = c^2 = e$.

          Let $f$ be an automorphism on $V_4$. Because $f$ is a homomorphism,
          $f(e) = e$, that is, $f$ fixes $e$.

          Because of this and because of the definition of $V_4$, any
          permutation on $\{a,b,c\}$ is an automorphism of which there are
          $\Big|\{a,b,c\}\Big|! = 6$.
        \item There is clearly one automorphism on $\mathbb{Z}_4$, that is the
          identity automorphism $\theta : \mathbb{Z}_4 \rightarrow
          \mathbb{Z}_4$ where $\theta(n) = n$ for all $n \in \mathbb{Z}_4$.
          There is also the automorphism $\vartheta : \mathbb{Z}_4
          \rightarrow\mathbb{Z}_4$, which for $n \in \mathbb{Z}_4$
          $$
          \vartheta(n) =
            \begin{cases}
              &\vartheta(0) = 0 \\
              &\vartheta(1) = 3 \\
              &\vartheta(2) = 2 \\
              &\vartheta(3) = 1 \\
            \end{cases}
          $$
      \end{enumerate}

  \end{enumerate}
\end{document}
