\documentclass[10pt]{article}
\usepackage{mathtools}
\usepackage{amsfonts}
\usepackage{pifont}
\usepackage{pgfplots}
\pgfplotsset{compat=newest}
\newcommand*{\dydx}[0]{\frac{\text{d}y}{\text{d}x}}
\newcommand*{\dy}[0]{\text{d}y}
\newcommand*{\dx}[0]{\text{d}x}
\newcommand*{\dv}[0]{\text{d}v}
\newcommand*{\D}[2]{\frac{\partial{#1}}{\partial{#2}}}%
\newcommand*{\df}[2]{\frac{\text{d}{#1}}{\text{d}{#2}}}%
\author{BM Corser}
\title{Calculus 2 Assignment 3}
\begin{document}
    \maketitle 
    \begin{enumerate}
        \item The variables separable, exact, homogeneous, first order ordinary
            differential equation 
            \begin{align}
                x^2 + y^2\dydx = 0
            \end{align}
            can be solved by
    \begin{enumerate}
        \item separation of variables; if we rewrite (1) as
            \begin{align*}
                y^2\dydx &= -x^2 \\
            \end{align*}
            and integrate both sides with respect to $x$
            \begin{align*}
                \int \left(y^2\dydx\right)\dx &= -\int x^2 \dx \\
                \int y^2\dy &= -\int x^2 \dx \\
                \frac{y^3}{3} &= c_1 - \frac{x^3}{3} \\
            \end{align*}
            then rearrange to make $y$ the subject
            \begin{align*}
                y^3 &= c_2 - x^3 \\
                y &= \sqrt[3]{c_2-x^3} \\
            \end{align*}
                is a general solution to (1), where $c_2 = 3c_1$.
            \item observing that if we consider (1) to be of the form $$M(x,y) +
                N(x,y)\dydx = 0$$ where $M = x^2$ and $N = y^2$ then $\D{M}{x} =
                \D{M}{y} = 0$ and as such (1) is exact. We can then say that
                (1) has a general solution of the form $f(x,y) = c$ such that
                $\D{f}{x} = M$ and $\D{f}{y} = N$. Integrating $M$ with respect
                to $x$ gives
                \begin{align}
                    f &= \frac{x^3}{3} + g(y)
                \end{align}
                and integrating (2) with respect to $y$ gives
                \begin{align*}
                    \D{f}{y} &= g'(y).
                \end{align*}
                We can therefore deduce that $$g'(y) = N = y^2$$ and $$g(y) = \int
                N \dy = \frac{y^3}{3}$$ so $$f(x,y) = \frac{x^3 + y^3}{3}.$$
                \begin{align*}
                    \frac{x^3 + y^3}{3} &= c_1 \\
                    y^3 &= c_2-x^3 \\
                    y &= \sqrt[3]{c_2-x^3} \\
                \end{align*}
                is therefore a general solution to (1), where $c_2 = 3c_1$.
            \item using the substitution $y = vx$ and $\dydx = x\df{v}{x} + v$.
                Now
                \begin{align*}
                    x^2 + (vx)^2(x\df{v}{x} + v) &= 0 \\
                    x^2 + v^2x^3\df{v}{x} + x^2v^3 &= 0 \\
                    1 + v^2x\df{v}{x} + v^3 &= 0 \\
                \end{align*}
                which is variables separable, so we can rearrange and integrate
                with respect to $x$ as follows
                \begin{align*}
                    1 + v^2x\df{v}{x} + v^3 &= 0 \\
                    \frac{1}{v^2} + x\df{v}{x} + v &= 0 \\
                    \frac{1 + v^3}{v^2} &= - x\df{v}{x}  \\
                    \frac{v^2}{v^3 + 1}\df{v}{x} &= - \frac{1}{x}  \\
                    \int \frac{v^2}{v^3 + 1}\dv &= - \int \frac{\dx}{x}  \\
                    \frac{1}{3} \ln(v^3 + 1) &= - \ln x + c_1\\
                    \ln(v^3 + 1) &= - 3 \ln x + c_1\\
                    &= \ln x^{-3} + c_1\\
                    \exp(\ln(v^3 + 1)) &= \exp(\ln x^{-3} + c_1) \\
                    v^3 + 1 &= x^{-3}c_2 \\
                    \frac{y^3}{x^3} + 1 &=  \\
                    y^3 + x^3 &= c_2 \\
                    y &= \sqrt[3]{c_2 - x^3}. \\
                \end{align*}
                Then $y = \sqrt[3]{c_2 - x^3}$ where $c_2 = e^{c_1}$.
        \end{enumerate}
        \item The non-linear, non-exact, first order differential equation
            \begin{align}
                y^2 + x^2\dydx = 0
            \end{align}
            can be solved by
            \begin{enumerate}
                \item separation of variables; if we rewrite (3) as
                    \begin{align*}
                        x^2\dydx &= -y^2 \\
                        \dydx &= -y^2x^{-2} \\
                        y^{-2}\dydx &= -x^{-2} \\
                    \end{align*}
                    and integrate both sides with respect to $x$
                    \begin{align*}
                        \int \left(y^{-2}\dydx\right)\dx &= -\int x^{-2} \dx \\
                        \int y^{-2}\dy &= -\int x^{-2} \dx \\
                         x^{-1} + y^{-1} &= c\\
                    \end{align*}
                \item using the substitution $y = vx$ and $\dydx = x\df{v}{x} +
                    v$, giving
                    \begin{align*}
                        v^2 + x\df{v}{x} + v &= 0 \\
                        x\df{v}{x} &= - (v^2+v) \\
                        \frac{1}{v^2 v}\df{v}{x} &= -\frac{1}{x}. \\
                    \end{align*}
                    Since
                    \begin{align*}
                        \frac{1}{v^2 v} &= \frac{1 + v - v}{v(v +1)} \\
                         &= \frac{1 + v - v}{v(v +1)} \\
                         &= \frac{1 + v}{v(v +1)} - \frac{v}{v(v +1)} \\
                         &= \frac{1}{v} - \frac{1}{v +1} \\
                    \end{align*}
                    it follows that
                    \begin{align*}
                        \int \frac{1}{v} \dv - \frac{1}{v +1} \dv &= - \int \frac{\dx}{x} \\
                        \ln v - \ln(v+1) &= -\ln x + c_1 \\
                        \ln\left(\frac{v}{v+1}\right) +\ln x &= c_1 \\
                        x\cdot\frac{v}{v+1} &= c_2 \\
                        \frac{yx}{y+x} &= c_2 \\
                    \end{align*}
                    and
                    \begin{align*}
                        \frac{y + x}{yx} &= c_3 \\
                        \frac{y}{yx} + \frac{x}{yx} &= c_3 \\
                        \frac{1}{x} + \frac{1}{y} &= c_3 \\
                        x^{-1} + y^{-1} &= c_3. \\
                    \end{align*}
            \end{enumerate}
        \item
            \begin{align}
                f_1(x) \cdot g_1(y) + f_2(x)\cdot g_2(y)\dydx = 0
            \end{align}
            \begin{enumerate}
                \item The equation (4) is of the general form
                    $$M_0(x,y) + N_0(x,y)\dydx = 0.$$
                    Let $\mu = \frac{1}{f_2(x)g_1(y)}$.
                    Multiplying both sides of (4) by $\mu$ gives
                    \begin{align*}
                        M_0(x,y) + N_0(x,y)\dydx &= 0 \\
                        \mu M_0(x,y) + \mu N_0(x,y) \dydx &= 0 \\
                        \frac{f_1(x)}{f_2(x)} + \frac{g_2(y)}{g_1(y)}\dydx &= 0 \\
                    \end{align*}
                    which is of the form $$M_1(x) + N_1(y)\dydx = 0$$ and as such 
                    $\D{N_1}{x} = \D{M_1}{y} = 0$ and the ODE is exact. This
                    demonstrates that $\mu$ is an integrating factor.
                \item Now, let $\mu = \frac{1}{x^2y^2}$ and multiply (3) by
                    $\mu$, resulting in
                    \begin{align}
                        \frac{1}{x^2} + \frac{1}{y^2} \dydx &= 0
                    \end{align}
                    Now (5) is of the form $$M(x) + N(y)\dydx = 0,$$ which
                    means  (as shown above) that it is exact. In this form we
                    can solve by finding a function $f(x,y) = c$ such that
                    $\D{f}{x} = M$ and $\D{f}{y} = N$. Now
                    \begin{align*}
                        \int \D{f}{x} \dx &= \int x^{-2} \dx \\
                        f &= -x^{-1} + g(y) \\
                        \D{f}{y} &= g'(y) = y^{-2}
                    \end{align*}
                    then $g(y) = \int y^{-2} \dy = -y^{-1}$
                    and $f = -x^{-1} -y^{-1} = c$.
            \end{enumerate}
        \item Let $t = x^{-1}$ and $y = f(t)$. Now, by the chain rule
            \begin{align*}
                \df{y}{x} &= \df{t}{x}\cdot\df{y}{t} \\
                          &= (-x^{-2})\cdot\df{y}{t} \\
            \end{align*}
            and by the product rule and chain rule
            \begin{align*}
                \df{^2y}{x^2} &= \df{^2t}{x^2}\cdot\df{y}{t} + \df{t}{x}\cdot\left(\df{^2y}{t^2}\cdot\df{t}{x}\right) \\
                &= 2x^{-3}\cdot\df{y}{t} + (-x^{-2})\cdot\left(\df{^2y}{t^2}\cdot(-x^{-2})\right) \\
                &= 2x^{-3}\cdot\df{y}{t} + x^{-4}\cdot\df{^2y}{t^2} \\
            \end{align*}
            then
            \begin{align*}
                x^4\cdot\df{^2y}{x^2} &= 2x\cdot\df{y}{t} + \df{^2y}{t^2} \\
            \end{align*}
            and
            \begin{align*}
                2x^3\cdot\df{y}{x} &= -2x\cdot\df{y}{t} \\
            \end{align*}
            so, under the substitution $x=t^{-1}$, $\df{^2y}{t^2} - 4y = 4$.
            This is a linear, non-homogeneous, second order, ordinary
            differential equation with constant coefficients. Its homogeneous
            part has characteristic polynomial $w^2 - 4 = 0$ which has roots
            $\alpha = -2$ and $\beta = 2$, so with $\alpha, \beta \in
            \mathbb{R}$ and $\alpha \neq \beta$, the general solution to the
            homogeneous part of the ODE is $y = Ae^{-2t} + Be^{2t}$ with $A, B
            \in \mathbb{R}$. A particular solution to the original ODE  can be
            found by assuming that $p = n$ with $n \in \mathbb{R}$ and
            considering $\df{^2p}{x^2} - 4p = 4$, hence a particular solution
            is $p = -1$ and the general solution is therefore
            $y = Ae^{-2x^{-1}} + Be^{2x^{-1}} - 1$.
        \item The equation we are trying to find has the form 
            \begin{align}
            P\df{^2y}{x^2} + Q\dydx + Ry = S(x)
            \end{align}
            where the $P$,$Q$ and $R$ terms are the
            homogeneous part yielding the complimentary function $C(x)$ and
            $S(x)$ is the non-homogeneous part yielding the particular integral
            $p(x)$. The general solution to (6) given in the question can be
            written in the form $y = C(x) + p(x)$. Let's consider the
            complimentary function first. The form of the general solution
            suggests that $C(x) = e^{2x}(A\cos(3x) + B\sin(3x))$. We will
            proceed under this assumption since it also gives us $p(x) = 4x^2 +
            x - 1$.  The form of $C(x)$ tells us that the auxiliary function
            $Pt^2 + Qt + R = 0$ has roots $2\pm 3i$. If a polynomial has roots
            $\pm a$ then $(x \pm a)$ are factors of that polynomial. As such 
            \begin{align*}
                0 &= (x - 2 - 3i)(x - 2 + 3i) \\
                &= x^2 - 4 x + 13
            \end{align*}
            and we infer $P = 1$, $Q = -4$, $R = 13$. Now we will
            consider the particular integral $p(x)$ and its relationship to
            $S(x)$. Because $p$ is a solution to (6), $ \df{^2p}{x^2} - 4\df{p}{x} +13p = S(x).$
            Now
            \begin{align*}
                p &= 4x^2 + x - 1 \\
                \df{p}{x} &= 8x + 1 \\
                \df{^2p}{x^2} &= 8 \\
            \end{align*}
            and
            \begin{align*}
                S(x) &= 8 - 4(8x + 1) + 13(4x^2 + x - 1) \\
                &= 52 x^2 - 19 x - 9
            \end{align*}
            so
            $$
            \df{^2y}{x^2} - 4\dydx + 13y = 52 x^2 - 19 x - 9
            $$

    \end{enumerate}
\end{document}
