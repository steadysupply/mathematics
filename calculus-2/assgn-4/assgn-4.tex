\documentclass[10pt]{article}
\usepackage{mathtools}
\usepackage{amsfonts}
\usepackage{pifont}
\usepackage{pgfplots}
\pgfplotsset{compat=newest}
\newcommand*{\dTdt}[0]{\frac{\text{d}T}{\text{d}t}}
\newcommand*{\dy}[0]{\text{d}y}
\newcommand*{\dx}[0]{\text{d}x}
\newcommand*{\dt}[0]{\text{d}t}
\newcommand*{\dT}[0]{\text{d}T}
\newcommand*{\D}[2]{\frac{\partial{#1}}{\partial{#2}}}%
\newcommand*{\df}[2]{\frac{\text{d}{#1}}{\text{d}{#2}}}%
\author{BM Corser}
\title{Calculus 2, Assignment 4}
\begin{document}
    \maketitle 
    \begin{enumerate}
        \item 
        \begin{enumerate}
            \item In this context, $k$ cannot be 0, since $k = 0$ implies there
                is no relationship between $\dTdt$ and $T_s - T$. This gives us
                $|k| > 0$. However, I don't see a colloquial reason for $k >
                0$, since either or both of $T_s$ and $T$ can be negative. I
                can see it's in some sense meaningless to take a negative
                factor, since if $a$ is a factor of $b$ then $-a$ is also a
                factor of $b$.

                ...

                After having found a value for $T$, on the other hand, I can
                see that a negative value of $k$ would lead to a situation
                where instead of $T$ approaching $T_s$ as $t$ approaches
                infinity, $T$ would also go to minus infinity (which breaks the
                model of reality).
            \item $\df{T}{t} = k(T_s - T)$ is a first order variables separable
                ordinary differential equation and as such we can write
                \begin{align*}
                    \int\frac{1}{T_s - T}\dT &= k\int \dt \\
                    k(t + c) &= -\ln(T_s - T) \\
                    e^{-k(t + c)} &= T_s - T \\
                    T &= T_s - e^{-k(t+c)}. \\
                \end{align*}
            If $T = T_0$ when $t = 0$ we can write
                \begin{align*}
                    T_0 &= T_s - e^{-kc} \\
                    e^{-kc} &= T_s - T_0 \\
                    c &= -\frac{\ln(T_s - T_0)}{k} \\
                \end{align*}
                and
                \begin{align*}
                    T &= T_s - e^{-k\left(t-\tfrac{\ln(T_s - T_0)}{k}\right)} \\
                      &= T_s - e^{\ln(T_s - T_0)-kt} \\
                      &= T_s - \frac{e^{\ln(T_s - T_0)}}{e^{kt}} \\
                      &= T_s - \frac{T_s - T_0}{e^{kt}}. \\
                \end{align*}
            \item We are given $T(0) = 37$ and $T_s = 24$. 
                \begin{enumerate}
                    \item Let the amount of time between death and discovery be
                        $A$, now $T(A) = 34$, $T(A + 30) = 32$ and
                    \begin{align*}
                        e^{kA} &= \frac{13}{10} \\
                        kA &= \ln\left(\tfrac{13}{10}\right), \\
                        e^{k(A+30)} &= \frac{13}{8} \\
                        kA+k30 &= \ln\left(\tfrac{13}{8}\right). \\
                    \end{align*}
                    Substituting our value for $kA$ into the second equation
                    \begin{align*}
                        k &= \frac{\ln\left(\tfrac{13}{8}\right) - \ln\left(\tfrac{13}{10}\right)}{30} \\
                        &= \frac{\ln\left(\tfrac{5}{4}\right)}{30}. \\
                    \end{align*}
                \item With a value for $k$, we can write
                        $$T = T_s - (T_s - T_0)\cdot\exp{\left(-\frac{A\ln\left(\tfrac{5}{4}\right)}{30}\right)}$$
                    and
                    \begin{align*}
                        34 &= 24 - (-13)\cdot\exp{\left(-\frac{A\ln\left(\tfrac{5}{4}\right)}{30}\right)} \\
                        \frac{10}{13} &= \exp{\left(-\frac{A\ln\left(\tfrac{5}{4}\right)}{30}\right)} \\
                        A &= -\frac{30\ln\left(\tfrac{10}{13}\right)}{\ln\left(\tfrac{5}{4}\right)} \\
                        &\approx 35.2729347...,
                    \end{align*}
                    which tells us the time of death was about 35 minutes before
                    high noon.
            \end{enumerate}
        \end{enumerate}
    \end{enumerate}
\end{document}
