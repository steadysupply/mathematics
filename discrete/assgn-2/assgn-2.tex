\documentclass[10pt]{article}
\usepackage{mathtools}
\usepackage{amsfonts}
\usepackage{pifont}
\newcommand*{\perm}[2]{{}^{#1}\!P_{#2}}%
\author{BM Corser}
\title{Discrete Assignment 2}
\begin{document}
    \maketitle 
    \begin{enumerate}
        \item Difference equations
        \begin{enumerate}
            \item Here, $u_n$ is an inhomogeneous first order difference
                equation, of the general form
                \begin{align*}
                    u_n = f(n)u_{n - 1} + g(n)
                        = U \cdot \prod_{i = 1}^n f(i) + \sum_{i = 1}^n\Bigg(g(i) \cdot \prod_{j = i + 1}^n f(j) \Bigg)
                \end{align*}
                where
                \begin{align*}
                    U &= 1 \\
                    f(n) &= 16^{n^3} \\
                    g(n) &= 2^{n^2(n + 1)^2}
                \end{align*}
                hence
                \begin{align*}
                    u_n = \prod_{i = 1}^n 16^{i^3} + \sum_{i = 1}^n\Bigg(2^{n^2(i + 1)^2} \cdot \prod_{j = i + 1}^n 16^{j^3} \Bigg)
                \end{align*}
            \item Let $k = n + 2$, then $a_{n + 2}$ can be written
                \begin{align*}
                    a_k = 4a_{k - 2} + 10 \cdot 3^{k - 2}
                \end{align*}
                Here, $a_k$ is an inhomogeneous second order difference
                equation with constant coefficients and can be written in the
                general form
                \begin{align*}
                    u_n = au_{n - 1} + bu_{n - 2} + f(n)
                \end{align*}
                where 
                \begin{align*}
                    a &= 0 \\
                    b &= 4 \\
                    f(n) &= 10 \cdot 3^{n - 2}
                \end{align*}
                and
                \begin{align*}
                    u_n = P(n) + G(n)
                \end{align*}
                where $P(n)$ is a particular solution to the inhomogeneous
                difference equation and $G(n)$ is a general solution to the
                homogenous part of the inhomogeneous difference equation.

                The homogenous part of the difference equation has
                characteristic polynomial $\lambda^2 - 4\lambda$ with distinct
                real zeros $w_1 = 2$ and $w_2 = -2$. There must be values $c_1$
                and $c_2$ such that the initial conditions are satisfied as
                follows
                \begin{align*}
                    c_1 + c_2 &= 9 \\
                    c_2 &= 9 - c_1 \\
                    \\
                    c_1w_1 + c_2w_2 &= 4 \\
                    2c_1 - 2c_2 &= \\
                    2c_1 - 2(9 - c_1) &= \\
                    2c_1 - 18 + 2c_1 &= \\
                    4c_1 &= 22 \\
                    c_1 &= 11 \\
                    \\
                    c_2 &= 9 - 11 \\
                    c_2 &= -2 \\
                \end{align*}

                so the general solution to the homogenous
                part is $G(n) = A\cdot11^n + B\cdot2^n$.

                $f(n)$ has the form $c\alpha^n$ where $\alpha = 3$ and $\alpha$
                is not a zero of the characteristic polynomial, therefore we
                can try $u_n = M\cdot3^n$ as a particular solution, hence

                \begin{align*}
                    M\cdot3^n &= 4M\cdot3^{n - 2} + 10 \cdot 3^{n - 2} \\
                    M\cdot3^2 &= 4M + 10 \\
                    9M &= 4M + 10 \\
                    5M &= 10 \\
                    M &= 2 \\
                \end{align*}
                and our particular solution is $P(n) = 2\cdot3^n$.
                Now we can write a general solution for the inhomogeneous
                difference equation $u_n$ as
                \begin{align*}
                    u_n &= P(n) + G(n) \\
                        &= A\cdot2^n + B\cdot2^n + 2\cdot3^n.
                \end{align*}
                The initial conditions $u_0 = 9$, $u_1 = 4$ imply that
                \begin{align*}
                    A + B + 2 &= 9 \\
                    A &= 7 - B  \\
                    \\
                    11A + 2B + 6 &= 4 \\
                    11(7 - B) + 2B &= 4 \\
                    9B &= 81 \\
                    B &= 9 \\
                    \\
                    A &= 7 - 9 \\
                    A &= 2 \\
                \end{align*}
                and therefore our general solution to the inhomogeneous
                difference equation $u_n = 2\cdot2^n + 9\cdot2^n + 2\cdot3^n$.
        \end{enumerate}
        \item The reproduction of flora on planet Zod can be described as a
            homogenous second order difference equation with constant
            coefficients and the general form 
            \begin{align*}
                u_n = au_{n - 1} + bu_{n - 2}
            \end{align*}
            where $a = 1$, $b = 6$ and $u_0 = U = 1$, $u_1 = V = 1$. Let $u_n = w^n$. We can now write
            \begin{align*}
                w^n &= aw^{n - 1} + bw^{n - 2} \\
                w^2 &= aw + b \\
                w^2 - aw - b &= 0. \\
            \end{align*}
            For the above to be true, $w$ must be a zero of the characteristic
            polynomial $\lambda^2 - a\lambda - b$. Since we know that $a^2 + 4b
            > 0$, this polynomial has distinct real zeros $w_1$ and $w_2$.
            Substituting $a$ and $b$ for their values and factorising gives the
            values of these zeros:
            $\lambda^2 - a\lambda - b = 0 = (\lambda - 3)(\lambda + 2)$

            We also know that for $c_1, c_2 \in \mathbb{R}$ we have $u_n =
            c_1w_1^n + c_2w_2^n$ with $c_1$ and $c_2$ taking values such that
            they satisfy the initial conditions $U$ and $V$.

            So, for $n = 0$
            \begin{align*}
                c_1 + c_2 = U &= 1 \\
                c_2 &= 1 - c_1\\
            \end{align*}
            and for $n = 1$
            \begin{align*}
                c_1w_1 + c_2w_2 &= V \\
                3c_1 - 2c_2 &= 1 \\
                3c_1 - 2(1 - c_1) &= 1 \\
                c_1 &= \frac{3}{5} \Longleftrightarrow c_2 = \frac{2}{5} \\
            \end{align*}
            then $u_n = 3(\tfrac{3}{5})^n - 2(\tfrac{2}{5})^n$.
        \item
            \begin{enumerate}
                \item When $u_0 = 0$, $u_n = 0$. Similarly, when $u_0 = 2$, $u_n = 2$.
                \item 
            \end{enumerate}
    \end{enumerate}
\end{document}
