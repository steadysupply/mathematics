\documentclass[10pt]{article}
\usepackage{mathtools}
\usepackage{amsfonts}
\usepackage{pifont}
\newcommand*{\perm}[2]{{}^{#1}\!P_{#2}}%
\author{BM Corser}
\title{Discrete Assignment 2}
\date{}
\begin{document}
    \maketitle 
    \begin{enumerate}
        \item Difference equations
        \begin{enumerate}
            \item Here, $u_n$ is an inhomogeneous first order difference
                equation, of the general form
                \begin{align*}
                    u_n = f(n)u_{n - 1} + g(n)
                        = U \cdot \prod_{i = 1}^n f(i) + \sum_{i = 1}^n\Bigg(g(i) \cdot \prod_{j = i + 1}^n f(j) \Bigg)
                \end{align*}
                where
                \begin{align*}
                    U &= 1 \\
                    f(n) &= 16^{n^3} \\
                    g(i) &= 2^{i^2(i + 1)^2}
                \end{align*}
                and
                \begin{align*}
                    \prod_{i = 0}^nf(i) &= \prod_{i = 0}^n16^{i^3} \\
                     &= 16^{\sum_{i = 0}^ni^3} \\
                     &= 16^{\frac{1}{4}n^2(n + 1)^2} \\
                     &= (2^4)^{\frac{1}{4}n^2(n + 1)^2} \\
                     &= 2^{n^2(n + 1)^2} \\
                     &= g(n)
                \end{align*}
                hence
                \begin{align*}
                    u_n &= g(n) + \sum_{i = 1}^n\Bigg(g(i) \cdot \prod_{j = i + 1}^n f(j) \Bigg) \\
                    &= g(n) + \sum_{i = 1}^n\Bigg(g(i) \cdot \frac{\prod_{j = 1}^n f(j)}{\prod_{k = 1}^i f(k)} \Bigg) \\
                    &= g(n) + \sum_{i = 1}^n\Bigg(g(i) \cdot \frac{g(n)}{g(i)} \Bigg) \\
                    &= (n + 1)\cdot 2^{n^2(n + 1)^2}
                \end{align*}
            \item Here, $a_n$ is an inhomogeneous second order difference
                equation with constant coefficients. We know that $u_n = G(n) +
                P(n)$ where $P(n)$ is a particular solution to the
                inhomogeneous difference equation and $G(n)$ is a general
                solution to the homogeneous part of the inhomogeneous
                difference equation and $G(n) + P(n)$ is a general solution to
                the inhomogeneous difference equation.

                The homogeneous part of the inhomogeneous difference equation
                has characteristic polynomial $\lambda^2 - 4$ with distinct
                real zeros $w_1 = 2$ and $w_2 = -2$, so the general solution to
                the homogeneous part is
                $$G(n) = A\cdot2^n + B\cdot(-2)^n.$$

                The inhomogeneous part has an $f(n)$ part of the form
                $c\alpha^n$ where $\alpha = 3$ and $\alpha$ is not a zero of
                the characteristic polynomial, therefore we can try $u_n =
                M\cdot3^n$ as a particular solution, hence

                \begin{align*}
                    M\cdot3^n &= 4M\cdot3^{n - 2} + 10 \cdot 3^{n - 2} \\
                    M\cdot3^2 &= 4M + 10 \\
                    9M &= 4M + 10 \\
                    5M &= 10 \\
                    M &= 2 \\
                \end{align*}
                and our particular solution is $P(n) = 2\cdot3^n$.
                Now we can write a general solution for the inhomogeneous
                difference equation $u_n$ as
                \begin{align*}
                    u_n &= G(n) + P(n) \\
                        &= A\cdot2^n + B\cdot(-2)^n + 2\cdot3^n.
                \end{align*}
                The initial conditions $u_0 = 9$, $u_1 = 4$ imply that
                \begin{align*}
                    A + B + 2 &= 9 \\
                    A &= 7 - B  \\
                    \\
                    2A - 2B &= -2 \\
                    A - B &= -1 \\
                    (7 - B) - B &= -1 \\
                    B &= 4  \Leftrightarrow A = 3 \\
                \end{align*}
                and therefore our general solution to the inhomogeneous
                difference equation is
                        $u_n = 3\cdot2^n - 4\cdot(-2)^n + 2\cdot3^n.$
            \item Here $b_n$ is an inhomogeneous second order difference
                equation with constant coefficients. We will solve it using the
                same technique employed above. The homogeneous part of $b_n$ has
                characteristic polynomial $\lambda^2 + 5 - 6\lambda = (\lambda - 3)^2 - 4$
                with zeros $w_1 = 5$ and $w_2 = 1$ and as such the general
                solution to the homogeneous part $G(n) = A\cdot5^n + B$.
                The inhomogeneous function is a polynomial of degree 1 in $n$
                and 1 is a zero of the characteristic polynomial (of
                multiplicity 1), so our particular solution will have the form
                $n(M_0 + nM_1) = n^2M_1 + nM_0$. So
                \begin{align*}
                    b_n &= 6b_{n - 1} -  5b_{n - 2} +120n - 33 \\
                    b_n - 6b_{n - 1} + 5b_{n - 2} &= 120n - 33 \\
                \end{align*}
                and
                \begin{align*}
                    nM_0 + n^2M_1 - 6(n - 1)M_0 - 6(n - 1)^2M_1 + 5(n - 2)M_0 + 5(n - 2)^2M_1 \\ = 120n - 33 \\
                    nM_0 + n^2M_1 6nM_0 + 6M0 - 6n^2M_1 + 12nM_1 - 6M_1 + 5nM_0 + 10M_0 + 5n^2M_1 - 20nM_1 \\ = 120n - 33 \\
                    n^2(M_1 - 6M1 + 5M_1) + n(M_0 - 6M_0 + 12M_1 + 5M_1 - 20M_1) - 4M_0 + 14M_1 \\ = 120n - 33 \\
                \end{align*}
                \begin{align*}
                    -8M_1 &= 120 \\
                    M_1 &= -15 \\
                    \\
                    -4M_0 + 14M_1 &= - 33 \\
                    M_0 &= - \tfrac{177}{4}. \\
                \end{align*}
                Then $G(n) + P(n) = A\cdot5^n + B -15n - \tfrac{177}{4}n^2$ and (using our values for $u_0$ and $u_1$)
                \begin{align*}
                    A &= B - 9 \\
                    30 &= 5A + B - 15 - \tfrac{177}{4} \\
                    A &= \tfrac{321}{16} \Leftrightarrow B = \tfrac{177}{16}
                \end{align*}
                and
                \begin{align*}
                    b_n = (\tfrac{321}{16})\cdot 5^n - \tfrac{177}{16} - 15n^2 - (\tfrac{177}{4})n \\
                \end{align*}
        \end{enumerate}
        \item The reproduction of flora on planet Zod can be described as a
            homogeneous second order difference equation with constant
            coefficients
            \begin{align*}
                u_n - u_{n - 1} - 6u_{n - 2} = 0,
            \end{align*}
            where $u_0 = u_1 = 1$. Let $g(x) = \sum_{i=0}^\infty u_ix^i$ be the
            generating function for the corresponding sequence
            $(u_i)_{i=0}^\infty$, then
            $$
            \begin{array}{rrllll}
                &       &N=0 &N=1 &N=2 \\
                \\
                &g(x) = &u_0 &+ u_1x &+ u_2x^2 &+ \cdots \\
                &-xg(x) =& &-u_0x &- u_1x^2 &+ \cdots \\
                &-6x^2g(x) =& &&-6u_0x^2 &+ \cdots .\\
            \end{array}
            $$
            The sum of the left hand sides of of the equations above is $(1 - x - 6x^2)g(x)$.
            Let the sum of the right hand sides of the equations above be $G$.
            We can see that for $N > 1$, the parts of G making up the coefficient of $x^N$
            take the same form as the difference equation (repeated up to $N$), so the coefficient of $x^N$
            will be 
            \begin{align*}
                u_N - u_{N - 1} - 6u_{N -2} = 0
            \end{align*}
            and we can therefore write
            \begin{align*}
                (1 - x - 6x^2)g(x) &= u_0 - x(u_1 - u_0)\\
                g(x) &= \frac{1}{1 - x - 6x^2} \\
                     &= -\frac{1}{(2x + 1)(3x  - 1)}
            \end{align*}
            and
            \begin{align*}
                \frac{1}{(2x + 1)(3x  - 1)} &= \frac{A}{2x + 1} + \frac{B}{3x - 1} \\
                                          1 &= A(3x - 1) + B(2x + 1) \\
            \end{align*}
            so $A = -\frac{2}{5}$, $B = \frac{3}{5}$ and
            \begin{align*}
                g(x) &= \frac{-\frac{2}{5}}{2x + 1} + \frac{\frac{3}{5}}{3x - 1} \\
                &= \frac{3}{5}(3x - 1)^{-1}-\frac{2}{5}(2x + 1)^{-1} \\
                &= \frac{3}{5}\sum_{i=0}^\infty (-3)^i x^i-\frac{2}{5}\sum_{i=0}^\infty (-2)^i x^i .\\
            \end{align*}
            The number of plants that the intergalactic botanist will have
            after $n$ years (from an initial crop of 3) will be the coefficient
            of $x^n$ in $3\cdot g(x) = 
                3\cdot \Big(\frac{3}{5}(-3)^n - \frac{2}{5}(-2)^n\Big)$.
        \item
            \begin{enumerate}
                \item When $u_0 = 0$, the sequence is constant $u_n = 0$. Similarly, when $u_0 = 2$, $u_n = 2$.
                \item 
                    \pagebreak
                \item 
                    \pagebreak
            \end{enumerate}
    \end{enumerate}
\end{document}
