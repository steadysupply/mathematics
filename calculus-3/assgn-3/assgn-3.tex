\documentclass[10pt]{article}
\usepackage{mathtools}
\usepackage{amsfonts}
\newcommand*{\La}{\mathcal{L}}
\newcommand*{\dxdt}[1]{\frac{\text{d}#1}{\text{d}t}}
\newcommand*{\nip}{\left(\tfrac{400}{\pi n}\right)}
\newcommand*{\dx}{\text{d}x}
\newcommand*{\px}[2]{\tfrac{\partial\dot #1}{\partial #2}}

\author{BM Corser}
\title{Calculus 3 Assignment 2}
\date{January 18, 2018}
\begin{document}
  \maketitle
  \begin{enumerate}
    \item Let $\dxdt{x} = y$, now
      \begin{align*}
        (t^2+1)\dxdt{y} - xty + x^2 &= x\cos t \\
        (t^2+1)\dxdt{y} &= x(\cos t + ty - x) \\
        \dxdt{y} &= x\left(\frac{\cos t + ty - x}{t^2+1}\right) \\
      \end{align*}
      let $z = t$. Now we can write the original differential equation as a
      system of first order autonomous differential equations
      \begin{align*}
        \dxdt{x} &= y \\
        \dxdt{y} &= x\left(\frac{\cos z + zy - x}{z^2+1}\right) \\
        \dxdt{z} &= 1.
      \end{align*}
      \pagebreak
    \item $\dot x = \alpha x^2 - x^4 = x^2(\alpha - x^2)$. The fixed points of
      this system are $x = 0$ and $x = \pm \sqrt{\alpha}$. To examine the
      behaviour of the sytem around these fixed points, we introduce an
      ``amount of change'' variable $\sqrt{\delta} \in \mathbb{R}$.

      In the case that $\alpha > 0$, when $x < -\sqrt{\alpha}$ we let
      $\sqrt{\delta} > 1$ and suppose that
      $x = \sqrt{\delta}(-\sqrt{\alpha}) = -\sqrt{\delta\alpha}$, now
      $x^2 = \delta\alpha > \alpha$, and the derivative of $x$,
      $\dot x = \delta\alpha(\alpha - \delta\alpha) < 0$.
      In the case $-\sqrt{\alpha} < x < 0$,
      let $0 < \sqrt{\delta} < 1$ and suppose $x = -\sqrt{\delta\alpha}$,
      now $x^2 = \delta\alpha < \alpha$ and
      $\dot x = \delta\alpha(\alpha-\delta\alpha) > 0$.
      In the case $0 < x < \sqrt{\alpha}$,
      let $0 < \sqrt{\delta} < 1$ and suppose $x = \sqrt{\delta\alpha}$,
      now $x^2 = \delta\alpha < \alpha$ and
      $\dot x = \delta\alpha(\alpha-\delta\alpha) > 0$.
      In the case $x > \sqrt{\alpha}$,
      let $\sqrt{\delta} > 1$ and suppose $x = \sqrt{\delta\alpha}$,
      now $x^2 = \delta\alpha > \alpha$ and
      $\dot x = \delta\alpha(\alpha-\delta\alpha) < 0$. We now have the
      necessary information to draw our phase portrait for the system when
      $\alpha > 0$, namely

      \bigskip
      \bigskip
      \bigskip
      \bigskip
      \bigskip
      \bigskip
      \bigskip
      \bigskip
      \bigskip
      \bigskip

      It is clear that the fixed point $x = -\sqrt{\alpha}$ is unstable, $x =
      0$ is semistable and $x = \sqrt{\alpha}$ is stable.

      In the case that $\alpha < 0$, since this dynamical system is one-dimensional, we know that $x \not \in \mathbb{C}$ so the only fixed point is $x = 0$.
      consider the behaviour of the dynamical system around the fixed points $x
      = \pm\sqrt{\alpha}$ and $x = 0$.
      When $x < -\sqrt{\alpha}$, we let
      $\sqrt{\delta} > 1$ and suppose that
      $x = -\sqrt{\delta\alpha}$, now
      $x^2 = \delta\alpha < \alpha$ (ie. a \emph{more} negative number) and
      $\dot x = \delta\alpha(\alpha - \delta\alpha) < 0$.
      In the case $-\sqrt{\alpha} < x < 0$,
      let $0 < \sqrt{\delta} < 1$ and suppose $x = -\sqrt{\delta\alpha}$,
      now $x^2 = \delta\alpha > \alpha$ and
      $\dot x = \delta\alpha(\alpha-\delta\alpha) > 0$.
      In the case $0 < x < \sqrt{\alpha}$,
      we
      let $0 < \sqrt{\delta} < 1$ and suppose $x = \sqrt{\delta\alpha}$,
      now $x^2 = \delta\alpha > \alpha$ and
      $\dot x = \delta\alpha(\alpha-\delta\alpha) > 0$.
      In the case $x > \sqrt{\alpha}$,
      let $\sqrt{\delta} > 1$ and suppose $x = \sqrt{\delta\alpha}$,
      now $x^2 = \delta\alpha < \alpha$ and
      $\dot x = \delta\alpha(\alpha-\delta\alpha) < 0$. We now have enough
      information to conclude that our first phase portrait equally for $\alpha
      > 0$ illustrates the case when $\alpha < 0$.

      Finally, considering the case when $\alpha = 0$, our dynamical system can
      be written $\dot x = -x^4$, but our phase portrait still provides a
      satisfactory illustration when the interval between
      $-\sqrt{\alpha}$ and $\sqrt{\alpha}$ is exactly zero.


      \item
        \begin{enumerate}
          \item $A = \begin{pmatrix} 3 & 0 \\ \beta & 3 \end{pmatrix}$, $\det(A
              - \lambda I) = (3 - \lambda)^2 = 0$, $\lambda = 3$.

            \item
              \begin{enumerate}
                  \item $\beta = 0$:
                    \begin{align*}
                      (A - \lambda I)\mathbf{x} &= \mathbf{0} \\
                      \begin{pmatrix} 0 & 0 \\ 0 & 0 \end{pmatrix}
                      \begin{pmatrix} x \\ y \end{pmatrix}
                        &=
                      \begin{pmatrix} 0 \\ 0 \end{pmatrix} \\
                    \end{align*}
                  Any value of $x$ and $y$ will satisfy this equation, so the eigenvectors are 
                      $\begin{pmatrix} 1 \\ 0 \end{pmatrix}$ and
                      $\begin{pmatrix} 0 \\ 1 \end{pmatrix}$. Because A is a
                        $2\times 2$ matrix and there are 2 independent
                        eigenvectors, $A$ is not defective when $\beta = 0$.
                  \item $\beta \not= 0$:
                    \begin{align*}
                      (A - \lambda I)\mathbf{x} &= \mathbf{0} \\
                      \begin{pmatrix} 0 & 0 \\ \beta & 0 \end{pmatrix}
                      \begin{pmatrix} x \\ y \end{pmatrix}
                        &=
                      \begin{pmatrix} 0 \\ 0 \end{pmatrix} \\
                      \begin{pmatrix} 0 \\ \beta x \end{pmatrix}
                        &=
                      \begin{pmatrix} 0 \\ 0 \end{pmatrix} \\
                    \end{align*}
                  To satisfy this equation $x$ must be equal to zero, so the only eigenvector of this $A$ is
                      $\begin{pmatrix} 0 \\ 1 \end{pmatrix}$.  Becaues there is
                        only one independent eigenvector, $A$ is defective when
                        $\beta \not= 0$.
              \end{enumerate}
      \item When $\beta = 0$ the system can be classified as an unstable star,
        where eigenvectors are the axes of the graph.

      \pagebreak

      When $\beta \not= 0$ the system can be classified as an unstable improper
      node where eigenvectors are the $y$ axis of the graph.

      \bigskip
      \bigskip
      \bigskip
      \bigskip
      \bigskip
      \bigskip
      \bigskip
      \bigskip
      \bigskip
      \bigskip
      \bigskip
      \bigskip
      \bigskip
      \bigskip
      \bigskip
      \bigskip
      \bigskip
      \bigskip
      \bigskip
      \bigskip
      \item Because the system has equal real eigenvalues, $\lambda_1 = \lambda_2 = \lambda$, the general form
        \begin{align*}
          \mathbf{x}(t) &= c_1e^{\lambda_1t}\mathbf{x}_1+c_2e^{\lambda_2t}\mathbf{x}_2 \\
                        &= e^{\lambda t}(c_1\mathbf{x}_1+c_2\mathbf{x}_2).
        \end{align*}
        Substituting values for the initial conditions $\mathbf{x}(t)$ at
        $t = 0$, $\lambda = 3$ and eigenvectors $\mathbf{x}_1$ and
        $\mathbf{x}_1$ we can write
        \begin{align*}
          \begin{pmatrix} 2 \\ -1 \end{pmatrix} &=
            e^{3 \cdot 0}\left(c_1\begin{pmatrix} 1 \\ 0 \end{pmatrix} +
              c_2\begin{pmatrix} 0 \\ 1 \end{pmatrix}\right) \\
                &= \begin{pmatrix} c_1 \\ 0 \end{pmatrix} +
              \begin{pmatrix} 0 \\ c_2 \end{pmatrix} \\
                &= \begin{pmatrix} c_1 \\ c_2 \end{pmatrix} \\
        \end{align*}
        therefore $c_1 = 2$ and $c_2 = -1$ and $\mathbf{x}(t) = e^{3t}\begin{pmatrix} 2 \\ -1 \end{pmatrix}$.
        \end{enumerate}
      \item The dynamical system in the question can be written
        \begin{align*}
          \dot x &= y \\
          \dot y &= -4\sin x - 3y.
        \end{align*}
        The fixed points of this system are those that satisfy
        $\dot x = \dot y = 0$. We can write $y = 0$ and see that
        $-4\sin x = 0$. The latter is true when $x$ takes values in
        $\{n\pi : n \in \mathbb{Z}\}$. Therefore the fixed points of this
        system are $\{(n\pi,0) : n \in \mathbb{Z}\}$.
        Computing the Jacobian of the system gives
        \begin{align*}
          J = \begin{pmatrix}
            \px{x}{x} & \px{x}{y} \\
            \px{y}{x} & \px{y}{y} \\
          \end{pmatrix} &=
          \begin{pmatrix}
            0 & 1 \\
            -4\cos x & -3 \\
          \end{pmatrix}.
        \end{align*}
      Given that values of $x$ are $n\pi$, we can see that there are two
      possible values for the bottom left entry $-4\cos x$, namely $4$ and
      $-4$. To classify fixed points relating to both possiblities, we will
      evaluate $J$ when $x = \pi$ and $x = 0$.

      Evaluating $J$ at the fixed point $(\pi,0)$ gives
        $
          J_\pi = \begin{pmatrix}
            0 & 1 \\
            4 & -3 \\
          \end{pmatrix}
        $.
        $$\det(J_\pi - \lambda I) = -\lambda(-3-\lambda) - 4 = (\lambda +
        4)(\lambda - 1)$$ tells us that the eigenvalues for $J_\pi$ are $\lambda
        = -4$ and $\lambda = 1$. For $\lambda = -4$,
        \begin{align*}
          (J_\pi = \lambda I)\mathbf{x} &= \mathbf{0} \\
          \begin{pmatrix}
            4 & 1 \\
            4 & 1 \\
          \end{pmatrix}
          \begin{pmatrix}
            x \\
            y \\
          \end{pmatrix}
          &=
          \begin{pmatrix}
            0 \\
            0 \\
          \end{pmatrix}
        \end{align*}
        and the eigenvectors associated with $\lambda = 4$ are of the form $
          \begin{pmatrix}
            4 \\
            -1 \\
          \end{pmatrix}$.
        For $\lambda = 1$,
        \begin{align*}
          (J_\pi = \lambda I)\mathbf{x} &= \mathbf{0} \\
          \begin{pmatrix}
            -1 & 1 \\
            4 & -1 \\
          \end{pmatrix}
          \begin{pmatrix}
            x \\
            y \\
          \end{pmatrix}
          &=
          \begin{pmatrix}
            0 \\
            0 \\
          \end{pmatrix}
        \end{align*}
        and the eigenvectors associated with $\lambda = 1$ are of the form $
          \begin{pmatrix}
            1 \\
            1 \\
          \end{pmatrix}$. The general form of the linearised system in the
          neighbourhood of point $(\pi, 0)$ is therefore
          $\mathbf{x}(t) = c_1e^{-4t}
          \begin{pmatrix}
            4 \\
            -1 \\
          \end{pmatrix}
          + c_2e^t$. Because eigenvalues are real and with opposite signs we
          can say that locally, the fixed point $(\pi, 0)$ is a saddle point
          with a trajectory inwards in the direction $
          \begin{pmatrix}
            4 \\
            -1 \\
          \end{pmatrix}
          $ and a trajectory outwards in the direction $
          \begin{pmatrix}
            1 \\
            1 \\
          \end{pmatrix}
          $.

      Evaluating $J$ at the fixed point $(0,0)$ gives
        $
          J_0 = \begin{pmatrix}
            0 & 1 \\
            -4 & -3 \\
          \end{pmatrix}
        $.
        $$\det(J_\pi - \lambda I) = -\lambda(-3-\lambda) + 4$$ doesn't
        factorise, which tells us $\lambda \in \mathbb{C}$. Applying the
        quadratic equation,
        $\lambda = \frac{-3\pm\sqrt{(-3)^2-4\cdot 1 \cdot 4}}{2 \cdot 1} = \tfrac{1}{2}(-3 \pm i\sqrt{7})$.
        We don't need to caculate the eigenvectors in this case because we know
        $(0, 0)$ behaves like a spiral locally. Because
        $\operatorname{Re}(\lambda) < 0$ we also know that this spiral is
        stable. As for the winding direction, because $\dot x = y$ and
        $\dot y = -3y$ when $x = 0$ so this spiral winds in a clockwise
        direction.

        We now have enough information to draw our phase portrait



  \end{enumerate}
\end{document}
