\documentclass[10pt]{article}
\usepackage{mathtools}
\usepackage{amsfonts}
\newcommand*{\La}{\mathcal{L}}
\newcommand*{\dxdt}[1]{\frac{\text{d}#1}{\text{d}t}}
\newcommand*{\nip}{\left(\tfrac{400}{\pi n}\right)}
\newcommand*{\dt}{\text{d}t}
\newcommand*{\px}[2]{\tfrac{\partial\dot #1}{\partial #2}}

\author{BM Corser}
\title{Calculus 3 Assignment 4}
\date{January 18, 2018}
\begin{document}
  \maketitle
  \begin{enumerate}
    \item
      \begin{enumerate}
        \item To show that $\bar\beta$ is independent of time, we must show
          $\dxdt{\bar\beta} = 0$. First note that, since $x = x_1 + x_2$,
          \begin{align*}
            \dot x = \dot x_1 + \dot x_2 
                   &= -\alpha y \frac{x_1}{x} -\alpha y \frac{x_2}{x} \\
                   &= -\alpha y \frac{1}{x}(x_1 + x_2) \\
                   &= -\alpha y
          \end{align*}
          also that, by the chain rule
          \begin{align*}
            \dxdt{(x^{-1})} = \dot x\left(-\frac{1}{x^2}\right)
                           = -\frac{\dot x}{x^2}
                           = \frac{\alpha y}{x^2} \\
          \end{align*}
          and that
          \begin{align*}
            \dxdt{(\beta_1x_1+\beta_2x_2)} &= \beta_1\dot x_1+\beta_2\dot x_2 \\
                           &= -\frac{\alpha y}{x}(\beta_1x_1+\beta_2x_2).
          \end{align*}
          By the product rule, then,
          \begin{align*}
            \dxdt{\bar\beta} &= \dxdt{(x^{-1})}(\beta_1x_1+\beta_2x_2) + (x^{-1})\dxdt{(\beta_1x_1+\beta_2x_2)} \\
                             &= \frac{\alpha y}{x^2}(\beta_1x_1+\beta_2x_2) + (x^{-1})(-\frac{\alpha y}{x}(\beta_1x_1+\beta_2x_2) \\
                             &= \frac{\alpha y}{x^2}(\beta_1x_1+\beta_2x_2) - \frac{\alpha y}{x^2}(\beta_1x_1+\beta_2x_2) \\
                             &= 0
          \end{align*}
          and $\bar\beta$ is independent of time. Now notice that 
          $\bar\beta = -\frac{\dot y}{x}$, $\bar\beta x = -\dot y$ and
          $\dot y = -\bar\beta x$. We have shown that $\bar\beta$ is constant
          and $\dot x = -\alpha y$, $\dot y = - \bar\beta x$, so by
          Lanchester's square law $\dxdt{(\alpha y^2 - \bar\beta x^2)}$ is also
          constant with respect to time (ie. constant throughout the battle).
          We can interpret $\bar\beta$ as the ``aggregate effectiveness'' of a
          soldier in army $X$.
        \item Since $\bar\beta$ is constant, we may evaluate it for any value
          of $t$ to obtain its quantity. Let $t=0$, now $\bar\beta = \tfrac{4
          \times 100 + 1 \times 500}{100 +500} = \tfrac{3}{2}$.
          Since $c = \alpha y^2 - \bar\beta x^2$ is constant,we may obtain a its
          value in a similar fashion, namely
          $c = (2 \times 1000^2) - (\tfrac{3}{2} \times 600^2) = 1 460 000$.
          Because $c > 0$, we know $Y$ is the victor, with $y =
          \sqrt{\tfrac{c}{\alpha}} \approx 854$ archers remaining (plus one
          poor guy missing $\tfrac{3}{5}$ths of  his body) when $x = 0$.
      \end{enumerate}
      \item
        \begin{enumerate}
          \item As above, to show $N$ is constant, we must show its derivative
            is zero with respect to time.
            \begin{align*}
              N &= S + I \\
                &= I + S \\
              \dot N &= \dot I + \dot S \\
                     &= \beta SI - \beta SI \\
                     &= 0
            \end{align*}
          \item ...

           \pagebreak

          \item By the definition of $\dot I$ and $N$, we may write
            \begin{align*} \label{I}
              \dot I &= -\beta SI \\
                     &= -\beta I(N - I) \\
              \frac{\dot I}{I(N - I)} &= -\beta.
            \end{align*}
            Since we know $\tfrac{1}{I(N - I)} \equiv \tfrac{X}{I} + \tfrac{Y}{N - I}$, we may write
            \begin{align*}
              1 &= X(N - I) + YI \\
                &= I(X - Y) + YN \\
            \end{align*}
            and infer by equating coefficients that
            $Y = \tfrac{1}{N}$ and $X = Y$. We may now write
            \begin{align*}
              \frac{\dot I}{I(N - I)} &= \frac{1}{N}\left(\frac{\dot I}{I} + \frac{\dot I}{N - I} \right) = -\beta \\
              \frac{\dot I}{I} + \frac{\dot I}{N - I} &= -\beta N \\
              \int \frac{\dot I}{I} \dt + \int \frac{\dot I}{N - I} \dt &= -\beta N \int 1 \dt. \\
            \end{align*}
            By substitution and laws of logarithms, let $A = e^c$ and write
            \begin{align*}
              \ln(I) + \ln(N -I) &= -\beta N t + c \\ 
              \ln(I(N -I)) &= \\
              I(N -I) &= Ae^{-\beta N t}.\\
            \end{align*}
            We can now see $SI = Ae^{-\tau}$. It is given that when $t = 0$, 
            $S = I$, so we know $A = I^2$. We can now write \begin{align*}
              SI &= I^2e^{-\tau} \\
                    e^\tau &= \frac{I}{S}. \\
            \end{align*}
            Finally, returning to $N = S + I$ and dividing by $I$, we can write
            \begin{align*}
              \frac{N}{I} &= \frac{S}{I} + 1 \\
              &= e^{-\tau} + 1 \\
              N &= I(e^{-\tau} + 1) \\
              I &= \frac{N}{e^{-\tau} + 1} \\
            \end{align*}
            and, dividing by $S$
            \begin{align*}
              \frac{N}{S} &=  1 + \frac{I}{S}\\
              &= 1 + e^\tau \\
              N &= S(1 + e^\tau) \\
              S &= \frac{N}{1 + e^{\tau}} \\
            \end{align*}
        \end{enumerate}
      \item There general form of the Lotka-Volterra equations is $\dot x =
        \alpha x - \beta x y$, $\dot y = \gamma xy - \delta y$. In this case it
        is given that $\alpha = 2, \beta = \tfrac{1}{2}, \gamma = 1$ and
        $\delta = \tfrac{1}{2}$.
        \begin{enumerate}
          \item Because we know the non-trivial fixed points of these equations
            are $(x_*, y_*) = \left(\frac{\delta}{\gamma},
            \frac{\alpha}{\beta}\right)$, we know there is a fixed point at $x
            = \frac{\delta}{\gamma} = 2$ and $y = \frac{\alpha}{\beta} = 4$,
            namely when there are 2000 rabbits and 4000 foxes.
          \item The Jacobian of the given system is
            \begin{align*}
              J &=
                \begin{pmatrix}
                  \px{x}{x} & \px{x}{y} \\ 
                  \px{y}{x} & \px{y}{y} \\ 
                \end{pmatrix} \\
                &=
                \begin{pmatrix}
                  2 - \tfrac{1}{2} x & -\tfrac{1}{2}x \\ 
                  \tfrac{1}{2} y & x - \tfrac{1}{2} \\ 
                \end{pmatrix} \\
            \end{align*}
            which, when we evalulate it at $x = y = 0$, gives $
              J_0 =
                \begin{pmatrix}
                  2 & 0 \\ 
                  0 & -\tfrac{1}{2} \\ 
                \end{pmatrix} 
              $.

              When $x = 10$ and $y = 10$, $\dot x = -30$ and $\dot x = 95$.
              Since we know $(2,4)$ is a fixed point, we know a prey population
              can support a predator population double its size, so it makes
              sense that having predator and prey populations equal will lead
              to predator population growth due to abundance of delicious
              rabbits. It also makes sense (in this situation) that prey
              population should be decreasing, because predators do not hunt
              according to the reproduction coefficients of their prey (ie.
              $\beta y > \alpha$).

         %    We can see that this matrix has eigenvalues $\lambda_1 = 2$
         %    and $\lambda_2 = -\frac{1}{2}$ with eigenvectors
         %    $\begin{pmatrix} 1 \\ 0 \end{pmatrix}$ and
         %    $\begin{pmatrix} 0 \\ 1 \end{pmatrix}$ respectively. Because the
         %      eigenvalues are of opposite signs, we know that $(0,0)$ is a
         %      saddle point and, due to association of eigenvalues with
         %      eigenvectors, has trajectories approaching the origin along the
         %      $y$-axis and heading away from the origin along the $x$-axis.

         %      This tells us that as the two species approach extinction,
         %      there reaches a point when the prey species with population
         %      $x$ will increase 
        \end{enumerate}
  \end{enumerate}
\end{document}
