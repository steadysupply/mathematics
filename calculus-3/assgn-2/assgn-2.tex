\documentclass[10pt]{article}
\usepackage{mathtools}
\usepackage{amsfonts}
\newcommand*{\La}{\mathcal{L}}
\newcommand*{\pinx}{\left(\tfrac{\pi nx}{400}\right)}
\newcommand*{\nip}{\left(\tfrac{400}{\pi n}\right)}
\newcommand*{\dx}{\text{d}x}

\author{BM Corser}
\title{Calculus 3 Assignment 2}
\date{January 18, 2018}
\begin{document}
  \maketitle
  \begin{enumerate}
    \item
    \begin{enumerate}
      \item $P_0$ is even, $P_1$ is odd, $P_2$ is even, $P_3$ is odd.
      \item Since for any odd function $f(x)$ we know that
        $\int^M_{-M}f(x)\text{d}x = 0$, all we need to demonstrate is any
        pairwise product $P_iP_j$ is odd. The product of two odd functions is
        odd and the product of an odd and an even function is also odd (ie.
        $\langle P_1, P_i\rangle = \langle P_3, P_i\rangle = 0$), so it only
        remains to prove $\langle P_0, P_2 \rangle = 0$, that is
        \begin{align*}
          \int^1_{-1}1\cdot\frac{1}{2}(3x^2 - 1)\text{d}x &= 0 \\
          \frac{1}{2}\left(\int^1_{-1}3x^2\text{d}x - \int^1_{-1}1\text{d}x\right) &= 0 \\
          \frac{1}{2}\left([x^3]^1_{-1} - [x]^1_{-1}\right) &= 0 \\
          \frac{1}{2}\left((1 -- 1)  - (1 -- 1)\right) &= 0 \\
          \frac{1}{2}\left(2 - 2\right) &= 0 \\
          \frac{1}{2}\left(0\right) &= 0 \\
          0 &= 0 \\
        \end{align*}
        QED
    \end{enumerate}
      \pagebreak
    \item
    \begin{enumerate}
      \item
        The definition of the Fourier series of $f(x)$ is
        \begin{align*}
          f(x) = a_0 + \sum^\infty_{n=1}\left(a_n\cos(\tfrac{\pi nx}{L}) + b_n\sin(\tfrac{\pi nx}{L})\right)
        \end{align*}
        where
        \begin{align*}
          a_0 &= \frac{1}{2L}\int^L_{-L}f(x)\text{d}x \\
          a_n &= \frac{1}{L}\int^L_{-L}f(x)\cos(\tfrac{\pi nx}{L})\text{d}x \\
          b_n &= \frac{1}{L}\int^L_{-L}f(x)\sin(\tfrac{\pi nx}{L})\text{d}x
        \end{align*}
        and here $f(x) = e^x$, $L = \pi$ so
        \begin{align*}
          a_0 &= \frac{1}{2\pi}\int^\pi_{-\pi}e^x\text{d}x \\
              &= \frac{1}{2\pi}(e^\pi + e^{-\pi}) \\
          a_n &= \frac{1}{\pi}\int^\pi_{-\pi}e^x\cos(nx)\text{d}x \\
              &= \frac{1}{\pi}\left[e^x\left(\frac{\cos(nx) + n\sin(nx)}{n^2+1}\right)\right]^\pi_{-\pi} \\
              &= \frac{1}{\pi}(e^\pi - e^{-\pi})\frac{(-1)^n}{n^2 + 1} \\
          b_n &= \frac{1}{\pi}\int^\pi_{-\pi}e^x\sin(nx)\text{d}x \\
              &= \frac{1}{\pi}\left[e^x\left(\frac{\sin(nx) - n\cos(nx)}{n^2+1}\right)\right]^\pi_{-\pi} \\
              &= -n\frac{1}{\pi}(e^\pi - e^{-\pi})\frac{(-1)^n}{n^2 + 1} \\
        \end{align*}
        and we can write the Fourier series as
        \begin{align*}
          f(x) &= \frac{1}{2\pi}(e^\pi + e^{-\pi}) + \sum^\infty_{n=1}\left(\frac{1}{\pi}(e^\pi - e^{-\pi})\frac{(-1)^n}{n^2 + 1}\cos(nx) -n\frac{1}{\pi}(e^\pi - e^{-\pi})\frac{(-1)^n}{n^2 + 1}\sin(nx)\right) \\
               &= \frac{1}{2\pi}(e^\pi + e^{-\pi}) + \frac{1}{\pi}(e^\pi - e^{-\pi})\sum^\infty_{n=1}\frac{(-1)^n}{n^2 + 1}\big(\cos(nx) -n\sin(nx)\big) \\
        \end{align*}
        and because $\sinh(\pi) = \frac{1}{2}(e^\pi + e^{-\pi})$ we can finally write
        \begin{align*}
          f(x) &= \frac{\sinh(\pi)}{\pi}\left(1 + 2\sum^\infty_{n=1}\frac{(-1)^n}{n^2 + 1}\big(\cos(nx) -n\sin(nx)\big)\right) \\
        \end{align*}
      \item
        Because $f(x)$ is continuous at $x = 0$, by Theorem 2.3, the Fourier
        series at $x = 0$ converges to $f(0) = e^0 = 1$ and we can write
        \begin{align*}
          1 &= f(0) \\
            &= \frac{\sinh(\pi)}{\pi}\left(1 + 2\sum^\infty_{n=1}\frac{(-1)^n}{n^2 + 1}\right) \\
          \frac{1}{\sinh(\pi)} &= \frac{1}{\pi}\left(1 + 2\sum^\infty_{n=1}\frac{(-1)^n}{n^2 + 1}\right) \\
          \text{cosech}(\pi) &=  \\
          \pi\text{cosech}(\pi) &=  1 + 2\sum^\infty_{n=1}\frac{(-1)^n}{n^2 + 1}. \\
        \end{align*}
        Because $f(x)$ has a discontinuity at $x = \pi$, the Fourier series at $x=\pi$ will converge to 
        $$\tfrac{1}{2}\left(\lim_{x \to \pi^+}f(x) + \lim_{x\to \pi}f(x)\right) = \tfrac{1}{2}(e^\pi + e^{-\pi}) = \cosh(\pi)$$
        now
        \begin{align*}
          \cosh(\pi) &= f(\pi) \\
          &= \frac{\sinh(\pi)}{\pi}\left(1 + 2\sum^\infty_{n=1}\frac{1}{n^2 + 1}\right) \\
          \frac{\cosh(\pi)}{\sinh(\pi)}&= \frac{1}{\pi}\left(1 + 2\sum^\infty_{n=1}\frac{1}{n^2 + 1}\right) \\
          \coth(\pi) &= \\
          \pi\coth(\pi) &= 1 + 2\sum^\infty_{n=1}\frac{1}{n^2 + 1} \\
        \end{align*}
    \end{enumerate}
      \item

    \begin{enumerate}
        \pagebreak
      \item The steady state solution $g(x)$ must be a linear function that
        satisfies the boundary conditions $g(0)  = 0$ and $g(400) = 200$. $g(x)
        = \tfrac{1}{2}x$ is such a function.
      \item $h(x) = u_0(x) - g(x) = \tfrac{3}{2}x - \tfrac{x^2}{400} -
        \frac{1}{2}x = x - \tfrac{x^2}{400}$. If $h(x)$ is even about $x = 200$
        then $h(200 + x) = h(200 + x)$ and
        \begin{align*}
          (200 + x) - \frac{(200 + x)^2}{400} &= (200 - x) - \frac{(200 - x)^2}{400} \\
          200 + x - \frac{200^2 + 400x + x^2}{400} &= 200 - x - \frac{200^2 - 400x + x^2}{400} \\
          400x - 200^2 + 400x + x^2 &= - 400x - 200^2 - 400x + x^2 \\
          x^2 - 200 &= x^2 - 200 \\
        \end{align*}
      \item $$u(x,t) = \sum^\infty_{n=1}B_n\sin\left(\tfrac{\pi nx}{L}\right)e^{\lambda^2_nt} + \frac{x}{2}$$
        where $L=400$, $\lambda_n = \alpha\tfrac{\pi n}{L}$, $\alpha^2 = 1$,
        \begin{align*}
          B_n &= \frac{2}{400}\int_0^{400}(x-\tfrac{x^2}{400})\sin\pinx\dx \\
              &= \frac{1}{200}\left(\Big[- (x - \tfrac{x^2}{400})\cos\pinx\nip\Big]_0^{400} + \int_0^{400}(1 - \tfrac{x}{200})\cos\pinx\nip\dx\right) \\
              &= \frac{2}{\pi n}\left(\int_0^{400}(1 - \tfrac{x}{200})\cos\pinx\dx\right) \\
              &= \frac{2}{\pi n}\left(\Big[(1 - \tfrac{x}{200})\nip\sin\pinx\Big]_0^{400} + \tfrac{1}{200}\nip\int_0^{400}\sin\pinx\dx\right) \\
              &= \frac{4}{(\pi n)^2}\left(\int_0^{400}\sin\pinx\dx\right) \\
              &= \frac{4}{(\pi n)^2}\Big[-\nip\cos\pinx\Big]_0^{400} \\
              &= -\frac{1600}{(\pi n)^3}\left(\cos(\pi n) - 1\right). \\
        \end{align*}
        We can now write
        \begin{align*}
          u(x,t) &= - 1600 \sum^\infty_{n=1}\frac{1}{(\pi n)^3}\big((-1)^n - 1\big)sin\big(\tfrac{\pi nx}{400}\big)e^{\tfrac{(\pi n)^2}{400^2}t} + \frac{x}{2} \\
        \end{align*}
    \end{enumerate}

  \end{enumerate}
\end{document}
