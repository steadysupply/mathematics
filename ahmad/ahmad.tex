
\documentclass[]{article}
\usepackage{amsmath, amssymb}
\begin{document}
                        The Diophantine Equation

\begin{equation}
	y(y+1)=x(x+1)(x+2)
\end{equation}

In this section we use the arithmetic of the cubic field
 \begin{equation}
	K=\mathbb{Q}(\theta), \theta -4\theta +2=0,
 \end{equation} 
 if we set :
      $X=2x+2$, $Y=2y+1$
     we will simplify $X=2x+2$ and $Y=2y+2$ to find $x$ , $y$ after that we will substitute in equation (1)we will get:
     \begin{equation}
     	2Y^2=X^3-4X+2
     \end{equation}

     Clearly any solution of equation (3) must have $X$ even and $Y$ odd . We will show that the only solution of equation (3) are 
     \\$(X,Y)=(-2,\pm1), (0,\pm1), (2,\pm1), (4,\pm5),(12,\pm29)$.
     \\We let $\theta, 	\theta', \theta'' \in \mathbb{C}$ be  the three roots of $x^3-4x+2=0$ so that 
     $x^3-4x+2=(x-\theta) (x-\theta') (x-\theta'').$ We will expand the right hand side and take  common factors of  $x$ and $x^2$. After that we will equate the coefficients of $x$, $x^2$ and for the constant term. These equalities can be written $\theta + \theta' +\theta'' = 0$ for $x^2$, $\theta\theta' + \theta' \theta'' + \theta''\theta = -4$ for $x$ and $\theta\theta'\theta'' = -2$ for the constant term. 
     \\the following solution in the source ().
     
 We will take another example from the source (An introduction to diophantine equation).
     \begin{equation}
  6x+10y-15z=1
     \end{equation}
     we have $y=1(mod 3)$, hence $y=1+3s$, s$\in \mathbb{Z}$.
     We will substitute $y$ in equation (4) the equation becomes
     $6x-15z=-9-30s$, or equivalently ,
     \begin{equation}
      2x-5z=-3-10s
      \end{equation}
  Because $z=1 (mod 2)$, $z=1+2t$, where $ t\in\mathbb{Z}$,we will substitute z in equation (5) the equation becomes $x=1-5s+5t$.Hence the solution are :
 \\ $ (x,y,z)=(1-5s+5t,1+3s,1+2t)$, $s,t\in \mathbb{Z}$.
 
  
     \newpage
     
     Prove that equation 
     $$x^3-x^2+8=y^2$$
     is not solvable in integer to solve this equation for $x$ odd , we will write the equation as 
     \\$(x+2)(x^2-2x+4)=x^2+y^2$.
     It is clear that $gcd(x,y)=1$. 
     the greatest common divisor (gcd) of two or more integers, which are not all zero, is the largest positive integer that divides each of the integers. 
     If $x=4k+1$, then $x+2=4k+3$ has a prime divisor of this form that divides $x^2+y^2$, impossible .If $x=4k+3$, then $x^2-2x+4$ is of the form $4m+3$, and by the same argument , we again get a contradiction. 
     \\ For $x=2u$ , the equation becomes 
     $$2u^3-u^2+2=z^2.$$
     If u is odd, then the left hand side is congruent to 3 (mod 4), and so it cannot be a perfect square . If u is even , then the left hand side is congruent to 2(mod4) and again cannot be a perfect square.\\
     \\We will take another example from source (Algebraic number theory and fermat's last theorem).
     $$x^2+7=2^n$$
    to solve this equation we work in $\mathbb{Q}(\sqrt{-7})$ whose ring of integers has unique factorization.\\
     unique factorization domain means (UFD) is an integral domain (a non-zero commutative ring in which the product of non-zero elements is non-zero) in which every non-zero non-unit element can be written as a product of prime elements (or irreducible elements), uniquely up to order and units, analogous to the fundamental theorem of arithmetic for the integers.
     For x is odd and we will suppose x is positive.
     Assume first that n is even we have factorization of integers:
     $$(2^{n/2}+x)(2^{n/2}-x) =7$$    
     so that $2^{n/2}+x=7$, $2{n/2}-xz=1$,
     
     so 
     $$2^{1+n/2}=8$$ 
     and n=4, x=3.
     Now let n be odd, and assume $n>3$.\\
 We have to use (Dedekind’s Theorem) to factorization into prime \\$$2=(1+\sqrt{-7}/2)(1-\sqrt{-7}/2)$$.
  Now let x is odd, $x=2k+1$, so $x^2+7=4k^2+4k+8$ is divisible by 4.Putting $m=n-2$, we can rewite the equation to be solved as 
  $$\frac{x^2+7} {4}=2^m$$     
  
  so that 
  $$(\frac{x+\sqrt{-7}} {2} )(\frac{x-\sqrt{-7}} {2})=(\frac{1+\sqrt{-7}} {2})^m (\frac{1-\sqrt{-7}} {2})^m$$
  where the right hand side is a prime factorization. Neither $(1+\sqrt{-7})/2$ nor $(1-\sqrt{-7})/2$ is a common factor of the terms on the left because such a factor would divide their difference, $\sqrt{-7}$, which is seen to be impossible by taking norms. Comparing the two factorizations, since the only units in the integers of $\mathbb{Q}(\sqrt{-7})$ are +1, we  must have 
  $$ \frac{x+\sqrt{-7}}{2}=+(\frac{1+\sqrt{-7}}{2})^m$$
  
  for which we derive 
  $$+\sqrt{-7}=(\frac{1+\sqrt{-7}}{2})^m - (\frac{1-\sqrt{-7}}{2})^m.$$ 
  we claim that the positive sign cannot occur. For, putting $(\frac{1+\sqrt{-7}}{2})^m=a$, $(\frac{1-\sqrt{-7}}{2})=b$ we have 
  $$a^m -b^m=a-b.$$
  Then $a^2\equiv(1-b)^2\equiv1$ (mod $b^2$)
  \\  since ab=2, and so 
\\  $a^m\equiv a(a^2)^\frac{m-1}{2}\equiv a $ (mod $b^2$)
 \\ where$a\equiv a-b$ (mod $b^2$),
 a contradiction.
 The only solution of the equation $x^2+7=2^n$ in integers x,n are: 
 \\x=1  3  5  11  181 
 \\n= 3  4  5  7  15
\\ We can find the rest of solution in the source (Algebraic number theory and fermat's last theorem).
\end{document}
