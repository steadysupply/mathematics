\documentclass[10pt]{article}
\usepackage{mathtools}
\usepackage{amsfonts}
\usepackage{enumerate}
\usepackage{verbatim}
\newcommand*{\La}{\mathcal{L}}
\author{BM Corser}
\title{Games, Choice and Optimisation Assignment 3}
\begin{document}
  \maketitle 
  \begin{enumerate} \item \begin{enumerate} \item Because the column Colin $B$
        dominates the column Colin $A$, we can rewrite the game

      \setlength{\unitlength}{1.0cm}
      $$ \begin{picture}(4,3)
        \put(2.5, 2.7){Colin}
        \put(2.4,2.1){\(B\)}
        \put(3.2,2.1){\(C\)}
        \put(3.8,2.1){\(D\)}
        \put(1.3,2){\line(1,0){3.1}}

        \put(0.3, 1.25){Rose}
        \put(1.5,1.5){\(A\)}
        \put(1.5,1.0){\(B\)}
        \put(2,0.7){\line(0,1){1.7}}

        \put(2.3,1.5){\(-2\)} \put(2.5,1.0){\(3\)}

        \put(3.2,1.5){\(4\)} \put(3.2,1.0){\(1\)}

        \put(3.6,1.5){\(-4\)} \put(3.8,1.0){\(5\)} \put(4.2,1.0){.}

      \end{picture} $$

      Because Rose has exactly two strategies, we can represent Colin's payoff
      for each of Rose's pure strategies as vertical lines and draw slopes
      between his payoffs that represent what his strategy should be in
      response to Rose's mixed strategies.

      \pagebreak

      Having drawn our picture, we can see that Colin should select
      a mixed strategy involving $B$, $C$ and $D$. Because Rose wants to
      minimise her loss, she will chose a mixed strategy where Colin's best
      response is a mixed strategy involving $B$ and $C$.

      Thus, the solution of the game can be realised from the $2 \times 2$
      subgame

      $$ \begin{picture}(4,3) \put(2.5, 2.7){Colin} \put(2.4,2.1){\(B\)}
        \put(3.2,2.1){\(C\)} \put(1.3,2){\line(1,0){2.6}}

        \put(0.3, 1.25){Rose} \put(1.5,1.5){\(A\)} \put(1.5,1.0){\(B\)}
        \put(2,0.7){\line(0,1){1.7}}

        \put(2.3,1.5){\(-2\)} \put(2.5,1.0){\(3\)}

        \put(3.2,1.5){\(4\)} \put(3.2,1.0){\(1\)}

        \put(4.2,1.0){.}

      \end{picture} $$

      Suppose that Colin plays $xB$, $(1-x)C$. His expected payoff if Rose
      plays $A$ is $x(-2) + (1-x)4 = 4 - 6x$ and if Rose plays $B$, $x3 + (1-x)
      =  2x + 1$. The value for $x$ that satisfies these expressions is
      $\tfrac{3}{8}$ and we can write Colin's equalising strategy
      $\tfrac{3}{8}B$, $\tfrac{5}{8}C$.

      Suppose Rose plays $yA$, $(1-y)B$. Her expected payoff when Colin plays
      $B$ is $y(-2) + (1-y)3 = 3 - 5y$ and when Colin plays $C$, and
      $y4 + (1-y) = 3y + 1$ when Colin plays $D$, so Rose's equalising strategy
      is $\tfrac{1}{4}A, \frac{3}{4}B$.

      The value of the game is $3 - 5 \cdot \tfrac{1}{4} = \tfrac{7}{4}$.

    \item The saddle points of $G$ are $G_{12}$, $G_{32}$, $G_{14}$, $G_{34}$
      and the value of the game is 1.

      \begin{comment}
      \end{comment}

      \item
        \begin{enumerate}[(i)]
          \item For Colin $C$ to dominate Colin $D$, since the values in the
            matrix represent Colin's loss, all entries in Colin $C$ must be less
            than or equal to corresponding entries in Colin $D$, and at least
            one entry must be strictly less that the corresponding entry in
            Colin $D$. This is the case when $-10 \leq x \leq - 4$. Notice,
            since $x$ cannot be both $-4$ and $-10$, this is not a strict
            inequality.
          \item For $H_{31}$ to be a saddle point, it must be less than or
            equal to each entry in its row and greater than or equal to each
            entry in its column. This is true when $-5 \leq x \leq -2$.
      \end{enumerate}
    \end{enumerate}
    \pagebreak
      \item
      \begin{enumerate}
        \item
          \begin{enumerate}[(i)]
            \item 
        The movement diagram of $G$ can be written

      $$
      \setlength{\unitlength}{1.0cm}
      \begin{picture}(6,4)

        \begin{comment}
        \put(0, 0){\line(0,1){0.3}}
        \put(0, 0){\line(1,0){0.3}}
        \put(6, 4){\line(0,-1){0.3}}
        \put(6, 4){\line(-1,0){0.3}}
        \end{comment}

        \put(3.5, 3.7){Colin}
        \put(2.4, 3.1){\(A\)}
        \put(3.2, 3.1){\(B\)}
        \put(3.8, 3.1){\(C\)}
        \put(4.4, 3.1){\(D\)}
        \put(5.0, 3.1){\(E\)}
        \put(1.3, 3.0){\line(1,0){4.2}}

        \put(0.3, 1.75){Rose}
        \put(1.5, 2.6){\(A\)}
        \put(1.5, 2.0){\(B\)}
        \put(1.5, 1.4){\(C\)}
        \put(1.5, 0.8){\(D\)}
        \put(2.0, 0.7){\line(0,1){3.0}}

        \begin{comment}
        \put(2.5, 2.7){\vector(1,0){2.6}}

        \put(2.5, 2.1){\vector(1,0){2.0}}
        \put(5.1, 2.1){\vector(-1,0){0.5}}

        \put(2.5, 1.5){\vector(1,0){1.5}}
        \put(5.1, 1.5){\vector(-1,0){1.0}}

        \put(2.5, 0.9){\vector(1,0){1.5}}
        \put(5.1, 0.9){\vector(-1,0){1.0}}

        \put(2.5, 2.7){\vector(0,-1){0.5}}
        \put(2.5, 0.9){\vector(0,1){1.1}}
        \end{comment}

      \end{picture}
      $$

            \item Start by ``shifting'' the game so that all payoffs are
              positive, we do this by adding 10 to all entries in the matrix.
              We can then write the linear programme
                $$
                \begin{array}{rrrrrrrrrr}
                  \text{maximise   }   &   x_1 &+& x_2 &+& x_3 &+& x_4, \\
                  \text{subject to   } &   14x_1 &+& 15x_2 &+& 8x_3 &+& 6x_4 &\geq& 1, \\
                                      &   8x_1 &+&13x_2 &+& 7x_3 &+& 18x_4 &\geq& 1, \\
                                      &   17x_1 &+& 14x_2 &+& 4x_3 &+& x_4 &\geq& 1, \\
                                      &   11x_1 &+& 8x_2 &+& 16x_3 &+& 7x_4 &\geq& 1, \\
                                      &   5x_1 &+& 9x_2 &+& 15x_3 &+& 12x_4 &\geq& 1, \\
                \end{array}
                $$
                $$x_1 \geq 0, x_2 \geq 0, x_3 \geq 0, x_4 \geq 0.$$
              \item By von Neumann's minmax theorem, any $m \times n$ game can
                be realised by a $k \times k$ subgame where
                $1 < k \leq \min(m,n)$.
                For the game $G$, $\min(m,n) = 4$.
          \end{enumerate}
        \item 
          \begin{enumerate}[(i)]
            \item An outcome of a game is a Nash equilibrium when neither
              player can independently change their strategy to increase their
              payoff. For the game $H$, the only entry satisfying this
              condition is $H_{21}$ (corresponding to Rose $A$ and Colin $B$).
            \item To find Colin's prudential strategy, we first write Colin's game

      \setlength{\unitlength}{1.0cm}
      $$ \begin{picture}(4,3) \put(2.5, 2.7){Colin} \put(2.4,2.1){\(A\)}
        \put(3.2,2.1){\(B\)} \put(1.3,2){\line(1,0){2.6}}

        \put(0.3, 1.25){Rose} \put(1.5,1.5){\(A\)} \put(1.5,1.0){\(B\)}
        \put(2,0.7){\line(0,1){1.7}}

        \put(2.5,1.5){\(1\)} \put(2.5,1.0){\(5\)}
        \put(3.2,1.5){\(5\)} \put(3.2,1.0){\(3\)}

        \put(4.2,1.0){.}

      \end{picture} $$
              Suppose Colin plays $xA, (1-x)B$ in this game. If Rose plays $A$,
              her expected payoff is $5 - 4x$ and if she plays $B$ it is
              $2x + 3$. The value of $x$ that satisfies both of these
              expression is $\tfrac{1}{3}$. Therefore Colin's equalising
              strategy is $\tfrac{1}{3}A, \tfrac{2}{3}B$ with security level
              $\tfrac{11}{3}$. When Colin plays his prudential strategy, Rose's
              expected payoff when she plays $A$ is $5$ and $\tfrac{8}{3}$ when
              she plays $B$. As such her counter-prudential strategy is $A$.
          \end{enumerate}
      \end{enumerate}
  \end{enumerate}
\end{document}
