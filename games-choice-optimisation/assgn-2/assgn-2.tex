\documentclass[10pt]{article}
\usepackage{mathtools}
\usepackage{amsfonts}
\usepackage{enumerate}
\newcommand*{\La}{\mathcal{L}}
\author{BM Corser}
\title{Games, Choice and Optimisation Assignment 2}
\date{January 18, 2018}
\begin{document}
  \maketitle 
  \begin{enumerate}
    \item
      \begin{enumerate}
      \item The dual of $\mathcal{L}$ is
        $$
        \begin{array}{rrcrcrl}
          \text{maximise   }   &   80y_1 &  + & 165y_2 & + & 105y_3, \\
          \text{subject to   } &   2y_1  &  + & 3y_2  & - & y_3    &\leq 6, \\
                               &   4y_1  &  + & y_2   & - & 2y_3   &\leq 2, \\
                               &  -3y_1  &  - & 2y_2  & + & 2y_3   &\leq -2, \\
        \end{array}
        $$
        $$y_1 \geq 0, y_2 \geq 0, y_3 \geq 0.$$
    \item
      \begin{enumerate}[(i)]
        \item $x_1 = 135$, $x_2 = 68$, $x_3 = 154$ is an optimal solution of
          $\mathcal{L}$ with value 638.
        \item The specified change to the objective function of $\mathcal{L}$
          would yield the following tableau
          $$
          \begin{array}{rrrrrr|r}
            x_1 & x_2 & x_3 & x_4 & x_5 & x_6 \\
            \hline
            1 & 0 & 0 & 0 & \tfrac{1}{2} & \tfrac{1}{2} & 135 \\
            0 & 0 & 1 & \tfrac{1}{5} & \tfrac{1}{5} & 1 & 154 \\
            0 & 1 & 0 & \tfrac{2}{5} & -\tfrac{1}{10} & \tfrac{1}{2} & 68 \\
            \hline
            0 & -4 & 0 & \tfrac{2}{5} & \tfrac{12}{5} & 2 & 638
          \end{array}
          $$
          All that remains is to send $r_4 \rightarrow r_4 + 4r_3$ to return
          $x_2$ to basis,
          $$
          \begin{array}{rrrrrr|r}
            x_1 & x_2 & x_3 & x_4 & x_5 & x_6 \\
            \hline
            1 & 0 & 0 & 0 & \tfrac{1}{2} & \tfrac{1}{2} & 135 \\
            0 & 0 & 1 & \tfrac{1}{5} & \tfrac{1}{5} & 1 & 154 \\
            0 & 1 & 0 & \tfrac{2}{5} & -\tfrac{1}{10} & \tfrac{1}{2} & 68 \\
            \hline
            0 & 0 & 0 & 2 & 2 & 4 & 910
          \end{array}
          $$
          this doesn't change the fact that this tableau shows
          an optimal solution, but changes the value of the linear programme to
          $910$.
        \item We first note that this linear programme is $\mathcal{L}$ with
          the addition of the variable $x_0$ as represented by the column
          $
          \begin{array}{r}
            x_0 \\
            \hline
            3  \\
            -3 \\
            1  \\
            \hline
            -1 \\
          \end{array}\
          $ in the initial tableau.
          In the final tableau for $\mathcal{L}$, $e_1 = \left(\begin{smallmatrix} 1 \\ 0 \\ 0 \end{smallmatrix}\right)$,
          $e_2 = \left(\begin{smallmatrix} 0 \\ 0 \\ 1 \end{smallmatrix}\right)$,
          $e_3 = \left(\begin{smallmatrix} 0 \\ 1 \\ 0 \end{smallmatrix}\right)$ and from the initial tableau
          $c = \left(\begin{smallmatrix} 6 \\ 2 \\ -2 \end{smallmatrix}\right)$, so
          $c^* = \left(\begin{smallmatrix} 6 \\ -2 \\ 2 \end{smallmatrix}\right)$ and
          $B = \left(\begin{smallmatrix} 0 & \tfrac{1}{2} & \tfrac{1}{2} \\ \tfrac{1}{5} & \tfrac{1}{5} & 1 \\ \tfrac{2}{5} & -\tfrac{1}{10} & \tfrac{1}{2} \end{smallmatrix}\right)$.

          We can now compute the new column corresponding to $x_0$ in the final tableau.

          \begin{align*}
          \begin{array}{c}
            B \left(\begin{smallmatrix} 3 \\ -3 \\ 1 \end{smallmatrix}\right)\\
            \hline
            (c^*)^TB\left(\begin{smallmatrix} 3 \\ -3 \\ 1 \end{smallmatrix}\right)-2\\
          \end{array}
            &=
          \begin{array}{c}
            \left(\begin{smallmatrix} -1 \\ 1 \\ 2 \end{smallmatrix}\right)\\
            \hline
            \left(\begin{smallmatrix} 6 & -2 & 2 \end{smallmatrix}\right) \left(\begin{smallmatrix} 3 \\ -3 \\ 1 \end{smallmatrix}\right)-2\\
          \end{array} \\
            &=
          \begin{array}{r}
            -1  \\
            1 \\
            2  \\
            \hline
            24 \\
          \end{array}.
          \end{align*}

          We can see the tableau still presents an optimal solution.

          $$
          \begin{array}{rrrrrrr|r}
            x_0 & x_1 & x_2 & x_3 & x_4 & x_5 & x_6 \\
            \hline
            -1 & 1 & 0 & 0 & 0 & 0.5 & 0.5 & 135 \\
            1 & 0 & 0 & 1 & 0.2 & 0.2 & 1 & 154 \\
            2 & 0 & 1 & 0 & 0.4 & -0.1 & 0.5 & 68 \\
            \hline
            24 & 0 & 0 & 0 & 0.4 & 2.4 & 2 & 638 \\
          \end{array} \\
          $$
          This final tableau shows an optimal solution with $x_1 = 135$, $x_2 = 68$, $x_3 = 154$ and a value of 638.
      \end{enumerate}
      \pagebreak
    \end{enumerate}
      \item
      \begin{enumerate}
        \item
          \begin{enumerate}[(i)]
          \item 
            \begin{enumerate}[(I)]
            \item Rose's choice $BC$ from $M$ does not satisfy the contraction
              condition because his choice from $AC$ does not contain $C$.
            \item Rose's choice $BC$ from $M$ does not satisfy the expansion
              condition because he chooses $D$ from all size 2 submenus
                containing $D$ but his choice from $M$ does not contain $D$.
          \end{enumerate}
          \item By the contraction condition, Colin did not choose $A$ because
            her choice $BD$ from $ABD$ does not contain $A$, she did not choose $C$
              because her choice $B$ from $BC$ doesn't contain $C$, similarly for
              $D$ because her choice $AC$ from $ACD$ doesn't contain $D$. For her
              choice to the satisfy expansion condition, the only element Colin
              must have chosen from $M$ is $B$. Hence the only reasonable
              element to be in $X$ is $B$.
        \end{enumerate}
          \item 
            \begin{enumerate}[(i)]
              \item $N$ is not a preference ordering because $A_1 \geq A_2$,
                $A_2 \geq A_3$, but $A_1 \not\geq A_3$.
                \item
                \begin{enumerate}[(I)]
                    \item The restriction of $\geq$ to $N_1$ is

                      $A_1 \geq A_1, A_7$

                      $A_3 \geq A_1, A_3$

                      $A_7 \geq A_3, A_7$.


                  The restriction of $\geq$ to $N_2$ is

                      $A_3 \geq A_3, A_4$

                      $A_4 \geq A_4, A_5$

                      $A_5 \geq A_3, A_5$

                      $A_7 \geq A_3, A_4, A_5, A_7$.

                    \item Mary chooses $A_\geq = \{A_2, A_6\}$ from $M$.

                    (A) No, since $N_{1\geq} = \{\}$.

                    (B) Yes, since $N_{2\geq} = \{A_7\}$. Mary chooses $A_7$.
                \end{enumerate}
              \item Because of the fact that Najma is indifferent between $X$ and $\tfrac{17}{6}W$, $\tfrac{17}{6}W$ we know
                  $U(X) = \tfrac{7}{16}\cdot U(W) + \tfrac{9}{16} \cdot U(W)$
                  and 

                  $U(W) = 20$, $U(Z) = 4$ we can write
                  $U(X) = \tfrac{7}{16}\cdot 20 + \tfrac{9}{16} \cdot 4 = 11$.
                  Similarly, since Najma is indifferent between $Y$ and $\tfrac{5}{8}X$, $\tfrac{3}{8}Z$ we know
                  $U(Y) = \tfrac{5}{8}\cdot U(X) + \tfrac{3}{8} \cdot U(Z) =
                  \tfrac{67}{8}$, to determine whether or not Najma prefers
                  $\tfrac{1}{5}W, \tfrac{4}{5}Y$ over $\tfrac{4}{5}X, \tfrac{1}{5}Z$ we evaluate the truth of the inequality
                  \begin{align*}
                    \tfrac{1}{5}W, \tfrac{4}{5}Y &\geq \tfrac{4}{5}X, \tfrac{1}{5}Z \\
                    \tfrac{1}{5}\cdot U(W) + \tfrac{4}{5}\cdot U(Y) &\geq \tfrac{4}{5}\cdot U(X) + \tfrac{1}{5}\cdot U(Z) \\
                    \tfrac{1}{5}\cdot 26 + \tfrac{4}{5}\cdot \tfrac{67}{8} &\geq \tfrac{4}{5}\cdot 11 + \tfrac{1}{5}\cdot 4 \\
                    \tfrac{107}{10} &\geq \tfrac{96}{10}, \\
                  \end{align*}
                  which is true, so Najma does prefer
                  $\tfrac{1}{5}W, \tfrac{4}{5}Y$ over $\tfrac{4}{5}X, \tfrac{1}{5}Z$.
            \end{enumerate}
      \end{enumerate}
  \end{enumerate}
\end{document}
